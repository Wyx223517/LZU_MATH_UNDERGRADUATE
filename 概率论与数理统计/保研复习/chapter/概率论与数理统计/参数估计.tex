\chapter{参数估计}\label{chap:参数估计}
\section{点估计}\label{sec:点估计}
\begin{definition}[点估计]\label{def:point_estimation}
设 $\mathbf{X} = (X_1, \dots, X_n)$ 为从某个总体中抽取的样本,$\hat{g}(\mathbf{X}) = \hat{g}(X_1, \dots, X_n)$ 是样本的函数,用 $\hat{g}(\mathbf{X})$ 作为 $g(\theta)$ 的估计,称为点估计 (point estimation)。
\end{definition}

\subsection{矩估计和最大似然估计}\label{subsec:矩估计和最大似然估计}
在 \ref{subsec:方差,相关系数和矩} 节中提到过矩的相关概念,这里我们将运用定义 \ref{def:sample_moments} 。

\begin{definition}[矩估计量与矩法]\label{def:moment_estimate}
设有总体分布族 $\{f(x, \theta), \theta \in \Theta\}$, $\theta$ 是参数空间,$g(\theta)$ 是定义在
$\Theta$ 上参数 $\theta$ 的函数,它可以表示为总体分布的某些矩的函数,即
\begin{equation}
g(\theta) = G(\alpha_1, \dots, \alpha_k; \mu_2, \dots, \mu_s).\label{eq:g_theta}
\end{equation}
设 $\mathbf{X} = (X_1, \dots, X_n)$ 是从上述分布族中抽取的简单样本,将式 (\ref{eq:g_theta}) 中的 $\alpha_i$ 和
$\mu_j$ 分别用它们 “自然” 的矩估计量 $a_{ni}$ 和 $m_{nj}$ 代替,得
\begin{equation}
\hat{g}(\mathbf{X}) = G(a_{n1}, \dots, a_{nk}; m_{n2}, \dots, m_{ns}).\label{eq:hat_g_X}
\end{equation}
则 $\hat{g}(\mathbf{X})$ 作为 $g(\theta)$ 的估计量,称为 $g(\theta)$ 的矩估计量 (moment estimate)。这种求矩
估计量的方法称为矩法 (moment method of estimation)。
\end{definition}
\begin{remark}
    这里就是用样本矩估计总体矩,重点是找到各阶矩与参数之间的关系。矩估计不唯一。
\end{remark}

极大似然估计是在参数分布族场合下常用的参数估计方法。

\begin{definition}[似然函数与对数似然函数]\label{def:likelihood_function}
设 $f(\mathbf{x}, \theta) = f(x_1, \dots, x_n, \theta)$ 为样本 $\mathbf{X} = (X_1, \dots, X_n)$ 的概率
函数。当 $\mathbf{x}$ 固定时,把 $f(\mathbf{x}, \theta)$ 看成 $\theta$ 的函数,称为似然函数 (likelihood function),
记为
\begin{equation}
L(\theta, \mathbf{x}) = f(\mathbf{x}, \theta), \quad \theta \in \Theta, \mathbf{x} \in \mathcal{X}. \label{eq:likelihood_L}
\end{equation}
其中 $\Theta$ 为参数空间,$\mathcal{X}$ 为样本空间。称 $\log L(\theta, \mathbf{x})$ 为对数似然函数,记为 $l(\theta, \mathbf{x})$。
\end{definition}

\begin{remark}
    似然函数和概率函数是同一表达式 \eqref{eq:likelihood_L},但表示两种不同含意。
当把 $\theta$ 固定,将其看成定义在样本空间 $\mathcal{X}$ 上的函数时,称为概率函数;当把 $\mathbf{x}$ 固
定,将其看成定义在参数空间 $\Theta$ 上的函数时,称为似然函数。这是两个不同的概念。
\end{remark}

\begin{definition}[极大似然估计]\label{def:MLE}
设 $\mathbf{X} = (X_1, \dots, X_n)$ 是从参数分布族 $\mathcal{F} = \{f(\mathbf{x}, \theta), \theta \in \Theta\}$ 中
抽取的简单随机样本,$L(\theta, \mathbf{x})$ 是似然函数,若存在统计量 $\hat{\theta}^* = \hat{\theta}^*(\mathbf{X})$,满足条件
\begin{equation}
L(\hat{\theta}^*, \mathbf{x}) = \sup_{\theta \in \Theta} L(\theta, \mathbf{x}), \quad \mathbf{x} \in \mathcal{X}, \label{eq:MLE_L}
\end{equation}
或等价地使得
\begin{equation}
l(\hat{\theta}^*, \mathbf{x}) = \sup_{\theta \in \Theta} l(\theta, \mathbf{x}), \quad \mathbf{x} \in \mathcal{X}, \label{eq:MLE_l}
\end{equation}
则称 $\hat{\theta}^*(\mathbf{X})$ 为 $\theta$ 的极大似然估计 (maximum likelihood estimation, MLE)。若待估
函数是 $g(\theta)$,则定义 $g(\hat{\theta}^*(\mathbf{X}))$ 为 $g(\theta)$ 的 MLE。
\end{definition}

\begin{remark}
    求极大似然估计,实际上是寻找极值.

通常 $\mathbf{X} = (X_1, \dots, X_n)$ 为独立同分布随机样本,每个 $X_i \sim f(x_i, \theta)$,则 $\mathbf{X} = (X_1, \dots, X_n)$ 的联合概率密度函数为 $\prod_{i=1}^n f(x_i, \theta)$。

似然函数为
$$L(\mathbf{x}, \theta) = \prod_{i=1}^n f(x_i, \theta)$$
对数似然函数为
$$\ln L(\mathbf{x}, \theta) = \sum_{i=1}^n \ln f(x_i, \theta)$$
因此,极大似然估计 $\hat{\theta}$ 的求解通常通过最大化对数似然函数得到:
$$\hat{\theta} = \operatorname{arg\,sup}_{\theta \in \Theta} \ln L(\mathbf{x}, \theta)$$

一般地,对 $\frac{\partial \ln L(\theta, \mathbf{x})}{\partial \theta}$ 求驻点,若Hessian矩阵 $H$ 在驻点处负定($H < 0$),则可判断该驻点为极大值点 $\hat{\theta}$。
\end{remark}

\subsection{点估计的优良性准则}\label{subsec:点估计的优良性准则}
\begin{definition}[无偏估计与渐近无偏估计]\label{def:unbiased_estimation}
设 $\mathbf{X} = (X_1, \dots, X_n)$ 为从总体 $\{f(\mathbf{x}, \theta), \theta \in \Theta\}$ 中抽取的样本,
$g(\theta)$ 是定义于参数空间 $\Theta$ 上的已知函数。$\hat{g}(\mathbf{X}) = \hat{g}(X_1, \dots, X_n)$ 是 $g(\theta)$ 的一个估
计量,如果
\begin{equation}
E_\theta(\hat{g}(\mathbf{X})) = g(\theta), \quad \theta \in \Theta, \label{eq:unbiased}
\end{equation}
则称 $\hat{g}(\mathbf{X})$ 为 $g(\theta)$ 的一个\textbf{无偏估计} (unbiased estimation)。记 $\hat{g}(\mathbf{X}) = \hat{g}_n(\mathbf{X})$,若
$E_\theta(\hat{g}_n(\mathbf{X})) \neq g(\theta)$, 但
\begin{equation}
\lim_{n\to\infty} E_\theta(\hat{g}_n(\mathbf{X})) = g(\theta), \quad \theta \in \Theta, \label{eq:asymptotically_unbiased}
\end{equation}
则称 $\hat{g}_n(\mathbf{X})$ 为 $g(\theta)$ 的\textbf{渐近无偏估计} (asymptotically unbiased estimation)。
\end{definition}

\begin{definition}[有效性]\label{def:efficient_estimator}
设 $\hat{g}_1(\mathbf{X}) = \hat{g}_1(X_1, \dots, X_n)$ 和 $\hat{g}_2(\mathbf{X}) = \hat{g}_2(X_1, \dots, X_n)$ 为 $g(\theta)$
的两个无偏估计量,若
\begin{equation}
D_\theta(\hat{g}_1(\mathbf{X})) \leq D_\theta(\hat{g}_2(\mathbf{X})), \quad \text{一切 } \theta \in \Theta, \label{eq:efficiency_condition}
\end{equation}
且至少存在一个 $\theta \in \Theta$,使得严格不等号成立,则称估计量 $\hat{g}_1(\mathbf{X})$ 比 $\hat{g}_2(\mathbf{X})$ 有效。
\end{definition}

\begin{definition}[相合估计]\label{def:consistent_estimation}
设对每个自然数 $n$, $\hat{g}_n(\mathbf{X}) = \hat{g}_n(X_1, \dots, X_n)$ 是 $g(\theta)$ 的一个估计
量。若 $\hat{g}_n(\mathbf{X})$ 依概率收敛到 $g(\theta)$,即对任何 $\theta \in \Theta$ 及 $\varepsilon > 0$ 有
\begin{equation}
\lim_{n\to\infty} P_\theta(|\hat{g}_n(\mathbf{X}) - g(\theta)| \geq \varepsilon) = 0, \label{eq:weakly_consistent}
\end{equation}
则称 $\hat{g}_n(\mathbf{X})$ 为 $g(\theta)$ 的\textbf{弱相合估计} (weakly consistent estimation)。若对任何 $\theta \in \Theta$ 有
\begin{equation}
P_\theta\left( \lim_{n\to\infty} \hat{g}_n(\mathbf{X}) = g(\theta) \right) = 1, \label{eq:strongly_consistent}
\end{equation}
则称 $\hat{g}_n(\mathbf{X})$ 为 $g(\theta)$ 的\textbf{强相合估计} (strongly consistent estimation)。若 $r > 0$ 和对
任何 $\theta \in \Theta$,有
\begin{equation}
\lim_{n\to\infty} E_\theta|\hat{g}_n(\mathbf{X}) - g(\theta)|^r = 0, \label{eq:r_consistent}
\end{equation}
则称 $\hat{g}_n(\mathbf{X})$ 为 $g(\theta)$ 的 $r$ \textbf{阶矩相合估计} (consistent estimation in $r$'th mean)。当
$r=2$ 时称为\textbf{均方相合估计} (consistent estimation in quadratic mean)。
\end{definition}
\begin{remark}
    分别对于依概率收敛 \ref{def:convergence_in_probability},以概率1收敛 \ref{def:convergence_almost_surely}和r阶收敛 \ref{def:r_order_convergence}。
\end{remark}
\subsection{一致最小方差无偏估计}\label{subsec:一致最小方差无偏估计}
\begin{definition}[均方误差]\label{def:MSE}
设 $\hat{g}(\mathbf{X})$ 为 $g(\theta)$ 的估计量,则称 $E_\theta(\hat{g}(\mathbf{X}) - g(\theta))^2$ 为 $\hat{g}(\mathbf{X})$ 的均
方误差 (mean square error, MSE)。
\end{definition}
\begin{align*}
E_\theta[\hat{g}(\mathbf{X}) - g(\theta)]^2 &= E_\theta[\hat{g}(\mathbf{X}) - E_\theta \hat{g}(\mathbf{X}) + E_\theta \hat{g}(\mathbf{X}) - g(\theta)]^2 \\
&= D_\theta(\hat{g}(\mathbf{X})) + [E_\theta \hat{g}(\mathbf{X}) - g(\theta)]^2 \\
&\quad + 2E_\theta[(\hat{g}(\mathbf{X}) - E_\theta \hat{g}(\mathbf{X}))(E_\theta \hat{g}(\mathbf{X}) - g(\theta))]
\end{align*}
第三项 $2E_\theta[(\hat{g}(\mathbf{X}) - E_\theta \hat{g}(\mathbf{X}))(E_\theta \hat{g}(\mathbf{X}) - g(\theta))]$ 展开后为 $2(E_\theta \hat{g}(\mathbf{X}) - g(\theta)) E_\theta(\hat{g}(\mathbf{X}) - E_\theta \hat{g}(\mathbf{X})) = 2(E_\theta \hat{g}(\mathbf{X}) - g(\theta)) (E_\theta \hat{g}(\mathbf{X}) - E_\theta \hat{g}(\mathbf{X})) = 0$。

故
$$E_\theta[\hat{g}(\mathbf{X}) - g(\theta)]^2 = D_\theta(\hat{g}(\mathbf{X})) + [E_\theta \hat{g}(\mathbf{X}) - g(\theta)]^2.$$
在 $E_\theta \hat{g}(\mathbf{X}) = g(\theta)$, $\theta \in \Theta$ 条件下考虑 MSE 最小会简化我们的讨论。

\begin{definition}[一致最小方差无偏估计 (UMVUE)]\label{def:UMVUE}
设 $\mathcal{F} = \{f(\mathbf{x}, \theta), \theta \in \Theta\}$ 是一个参数分布族,其中 $\Theta$ 为参数空
间,$g(\theta)$ 为定义在 $\Theta$ 上的可估函数。设 $\hat{g}^*(\mathbf{X}) = \hat{g}^*(X_1, \dots, X_n)$ 为 $g(\theta)$ 的一个无
偏估计,若对 $g(\theta)$ 的任一无偏估计 $\hat{g}(\mathbf{X}) = \hat{g}(X_1, \dots, X_n)$,都有
\begin{equation}
D_\theta(\hat{g}^*(\mathbf{X})) \leq D_\theta(\hat{g}(\mathbf{X})), \quad \text{一切 } \theta \in \Theta, \label{eq:UMVUE_condition}
\end{equation}
则称 $\hat{g}^*(\mathbf{X})$ 是 $g(\theta)$ 的一致最小方差无偏估计 (uniformly minimum variance unbiased
estimation, UMVUE)。
\end{definition}

\begin{lemma}[改进无偏估计的方法]\label{lem:3.4.1_RaoBlackwell}
    设 $T=T(\mathbf{X})$ 是一个充分统计量,而 $\hat{g}(\mathbf{X})$ 是 $g(\theta)$ 的一个无偏估计,则
$$h(T) = E_\theta(\hat{g}(\mathbf{X})|T)$$
是 $g(\theta)$ 的无偏估计,并且
\begin{equation}
D_\theta(h(T)) \leq D_\theta(\hat{g}(\mathbf{X})), \quad \text{一切 } \theta \in \Theta. \label{eq:Rao_Blackwell_inequality}
\end{equation}
其中等号当且仅当 $P_\theta(\hat{g}(\mathbf{X}) = h(T)) = 1$, 即 $\hat{g}(\mathbf{X}) = h(T)$, a.s. $P_\theta$ 成立。
\end{lemma}
\begin{proof}
    因 $T(\mathbf{X})$ 为充分统计量,按定义,给定 $T$ 时 $\mathbf{X}$ 的条件分布与 $\theta$ 无关。因此给定 $T$ 时,$\hat{g}(\mathbf{X})$ 的条件期望 $E_\theta(\hat{g}(\mathbf{X})|T)$ 与 $\theta$ 无关。所以 $h(T)$ 是统计量,可作为 $g(\theta)$ 的估计量,且有
\begin{equation}
E_\theta(h(T)) = E_\theta[E_\theta(\hat{g}(\mathbf{X})|T)] = E_\theta(\hat{g}(\mathbf{X})) = g(\theta), \quad \text{一切 } \theta \in \Theta. \label{eq:h_T_unbiased}
\end{equation}
因此 $h(T)$ 是 $g(\theta)$ 的无偏估计。为证式 (\ref{eq:Rao_Blackwell_inequality}),易知
\begin{align*}
D_\theta(\hat{g}(\mathbf{X})) &= E_\theta\{[\hat{g}(\mathbf{X}) - h(T)] + [h(T) - g(\theta)]\}^2 \\
&= E_\theta[\hat{g}(\mathbf{X}) - h(T)]^2 + E_\theta[h(T) - g(\theta)]^2 \\
&\quad + 2E_\theta\{[\hat{g}(\mathbf{X}) - h(T)][h(T) - g(\theta)]\}
\end{align*}
由于 $E_\theta[\hat{g}(\mathbf{X})|T] - h(T) = 0$, 可知
\begin{align*}
E_\theta\{[h(T) - g(\theta)] [\hat{g}(\mathbf{X}) - h(T)]\} &= E_\theta\{ E_\theta[(h(T) - g(\theta)) (\hat{g}(\mathbf{X}) - h(T))|T] \} \\
&= E_\theta\{ (h(T) - g(\theta)) [E_\theta(\hat{g}(\mathbf{X})|T) - E_\theta(h(T)|T)] \} \\
&= E_\theta\{ (h(T) - g(\theta)) [h(T) - h(T)] \} \\
&= E_\theta\{ (h(T) - g(\theta)) \cdot 0 \} = 0.
\end{align*}
故有
\begin{align*}
D_\theta(\hat{g}(\mathbf{X})) &= E_\theta[\hat{g}(\mathbf{X}) - h(T)]^2 + D_\theta(h(T)) \\
&\ge D_\theta(h(T)), \quad \text{一切 } \theta \in \Theta,
\end{align*}
且等号成立的充要条件是
$$E_\theta[\hat{g}(\mathbf{X}) - h(T)]^2 = 0,$$
此即 $\hat{g}(\mathbf{X}) = h(T)$, a.s. $P_\theta$ 成立。引理证毕。
\end{proof}
\begin{remark}
    这里其实用到了定义 \ref{def:回归} 的性质,本质上是随机变量在Hilbert空间
    \begin{equation}\label{eq:Hilbert Space}
        L(x) = \{\varphi(x): \varphi(x)\text{ is measurable}, \E\varphi(x) = 0, \E\varphi^{2}(x)<\infty\}
    \end{equation}
    上的投影,使得
    \begin{equation}
        \E(Y - \varphi(x))^{2} = \text{min!}
    \end{equation}
    这里的$\varphi(x) = \E( Y | X)$,为条件期望。
\end{remark}
\begin{remark}
    一般求UMVUE的方法有\textbf{充分完全统计量法}和\textbf{零无偏估计法}。
\end{remark}

\begin{theorem}[Lehmann-Scheff 定理]\label{thm:Lehmann_Scheff}
    设 $\mathbf{X} \sim \{f(\mathbf{x}, \theta), \theta \in \Theta\}$ 为参数空间。
令 $X_1, \dots, X_n$ 为从总体 $\mathbf{X}$ 中抽取的简单样本,$g(\theta)$ 为定义于参数空间 $\Theta$ 上的可估函数。设 $T(\mathbf{X})$ 为一个充分完全统计量,若 $\hat{g}(T(\mathbf{X}))$ 为 $g(\theta)$ 的一个无偏估计,则 $\hat{g}(T(\mathbf{X}))$ 是 $g(\theta)$ 的唯一的 UMVUE。
\end{theorem}
\begin{proof}
    先证唯一性。设 $\hat{g}_1(T(\mathbf{X}))$ 为 $g(\theta)$ 的任一无偏估计,令 $\delta(T(\mathbf{X})) = \hat{g}(T(\mathbf{X})) - \hat{g}_1(T(\mathbf{X}))$,则 $E_\theta \delta(T(\mathbf{X})) = E_\theta \hat{g}(T(\mathbf{X})) - E_\theta \hat{g}_1(T(\mathbf{X})) = 0$, $\theta \in \Theta$。由 $T(\mathbf{X})$ 为完全统计量,可知 $\delta(T(\mathbf{X})) = 0$, a.s. $P_\theta$ 成立,即 $\hat{g}(T(\mathbf{X})) = \hat{g}_1(T(\mathbf{X}))$, a.s. $P_\theta$ 成立,故唯一性成立。

再证一致最小方差性。设 $\varphi(\mathbf{X})$ 为 $g(\theta)$ 的任一无偏估计。令 $h(T(\mathbf{X})) = E_\theta(\varphi(\mathbf{X})|T)$,由 $T(\mathbf{X})$ 为充分统计量,故知 $h(T(\mathbf{X}))$ 与 $\theta$ 无关,是统计量。由引理 \ref{lem:3.4.1_RaoBlackwell} 可知
$$E_\theta(h(T(\mathbf{X}))) = g(\theta), \quad \text{一切 } \theta \in \Theta,$$
$$D_\theta(h(T(\mathbf{X}))) \leq D_\theta(\varphi(\mathbf{X})), \quad \text{一切 } \theta \in \Theta.$$
由唯一性得 $\hat{g}(T(\mathbf{X})) = h(T(\mathbf{X}))$, a.e. $P_\theta$ 成立,由上式可知
$$D_\theta(\hat{g}(T(\mathbf{X}))) = D_\theta(h(T(\mathbf{X}))) \leq D_\theta(\varphi(\mathbf{X})), \quad \text{一切 } \theta \in \Theta,$$
所以 $\hat{g}(T(\mathbf{X}))$ 为 $g(\theta)$ 的 UMVUE,且唯一。
\end{proof}
\begin{remark}
    可以总结找UMVUE的步骤:
    \begin{enumerate}[(1)]
        \item 找充分完全统计量 $T$
        \item 找 $T$ 的函数 $\hat{g}(T)$, s.t. $E_\theta \hat{g}(T) = g(\theta)$.
        或找 $\varphi(\mathbf{X})$, s.t. $E_\theta \varphi(\mathbf{X}) = g(\theta)$, $h(T) = E_\theta(\varphi(\mathbf{X})|T)$ 为 UMVUE。
    \end{enumerate}
\end{remark}

\begin{corollary}\label{cor:3.4.2}
    设样本 $\mathbf{X} = (X_1, \dots, X_n)$ 的分布为指数族
$$f(\mathbf{x}, \theta) = C(\theta) \exp\left\{\sum_{j=1}^k \theta_j T_j(\mathbf{x})\right\} h(\mathbf{x}), \quad \theta = (\theta_1, \dots, \theta_k) \in \Theta^*.$$
令 $T(\mathbf{X}) = (T_1(\mathbf{X}), \dots, T_k(\mathbf{X}))$。若自然参数空间 $\Theta^*$ 作为 $R_k$ 的子集有内点,且
$h(T(\mathbf{X}))$ 为 $g(\theta)$ 的无偏估计,则 $h(T(\mathbf{X}))$ 为 $g(\theta)$ 的唯一的 UMVUE。
\end{corollary}

\begin{theorem}[零无偏估计法]\label{thm:3.4.1}
    设 $\hat{g}(\mathbf{X})$ 是 $g(\theta)$ 的一个无偏估计,$D_\theta(\hat{g}(\mathbf{X})) < \infty$, 对任何 $\theta \in \Theta$。
若对任何满足条件 “$E_\theta l(\mathbf{X})=0$, 对一切 $\theta \in \Theta$” 的统计量 $l(\mathbf{X})$, 必有
\begin{equation}
\operatorname{Cov}_\theta(\hat{g}(\mathbf{X}), l(\mathbf{X})) = E_\theta[\hat{g}(\mathbf{X}) \cdot l(\mathbf{X})] = 0, \quad \text{一切 } \theta \in \Theta, \label{eq:covariance_condition}
\end{equation}
则 $\hat{g}(\mathbf{X})$ 是 $g(\theta)$ 的 UMVUE。
\end{theorem}


\begin{remark}
    从形式上看,条件 “$E_\theta l(\mathbf{X})=0, \theta \in \Theta$” 可解释为 “$l(\mathbf{X})$ 是零的无
偏估计”,由此得到求 UMVUE 的方法之一的名称 “零无偏估计法”。
\end{remark}

\subsection{Cramer-Rao不等式}\label{subsec:C-R inequality}
\begin{definition}[C-R 正则分布族]\label{def:CR_regular_family}
若单参数概率函数族 $\mathcal{F}=\{f(x, \theta), \theta \in \Theta\}$ 满足下列条件:
\begin{enumerate}
    \item 参数空间 $\Theta$ 是直线上的某个开区间;
    \item 对任何 $x \in \mathcal{X}$ 及 $\theta \in \Theta$, $f(x, \theta)>0$,即分布族具有共同支撑;
    \item 对任何 $x \in \mathcal{X}$ 及 $\theta \in \Theta$, $\frac{\partial f(x, \theta)}{\partial \theta}$ 存在;
    \item 概率函数 $f(x, \theta)$ 的积分与微分运算可交换,即
    $$\frac{\partial}{\partial \theta}\int f(x, \theta)dx = \int \frac{\partial}{\partial \theta}f(x, \theta)dx,$$
    若 $f(x, \theta)$ 为离散随机变量的概率分布,上述条件改为无穷级数和微分运算可交换;
    \item 下列数学期望存在,且
    $$0 < I(\theta) = E_\theta\left[\frac{\partial \log f(\mathbf{X}, \theta)}{\partial \theta}\right]^2 < \infty,$$
\end{enumerate}
则称该分布族为 C-R 正则分布族。其中 1.-5. 称为 C-R 正则条件。$I(\theta)$ 称为该分布的 Fisher 信息量(或称为 Fisher 信息函数)。
\end{definition}

\begin{theorem}[C-R不等式]\label{thm:Cramer_Rao_inequality}
    设 $\mathcal{F}=\{f(\mathbf{x}, \theta), \theta \in \Theta\}$ 是 C-R 正则分布族,$g(\theta)$ 是定义于参数
空间 $\Theta$ 上的可微函数。设 $\mathbf{X}=(X_1, \dots, X_n)$ 是由总体 $f(\mathbf{x}, \theta) \in \mathcal{F}$ 中抽取的简单
随机样本,$\hat{g}(\mathbf{X})$ 是 $g(\theta)$ 的任一无偏估计,且满足下列条件:
\begin{enumerate}
    \item 积分 
    $$\int \dots \int \hat{g}(\mathbf{x})f(\mathbf{x}, \theta)d\mathbf{x}$$
    可在积分号下对 $\theta$ 求导数,此处 $d\mathbf{x} = dx_1 \dots dx_n$, 则有
\end{enumerate}
\begin{equation}
D_\theta[\hat{g}(\mathbf{X})] \ge \frac{(g'(\theta))^2}{nI(\theta)}, \quad \text{一切 } \theta \in \Theta. \label{eq:Cramer_Rao_bound_general}
\end{equation}
\end{theorem}
\begin{remark}
    特别当 $g(\theta) = \theta$ 时,式 (\ref{eq:Cramer_Rao_bound_general}) 变为
\begin{equation}
D_\theta[\hat{g}(\mathbf{X})] \ge \frac{1}{nI(\theta)}, \quad \text{一切 } \theta \in \Theta. \label{eq:Cramer_Rao_bound_theta}
\end{equation}
当 $f(\mathbf{x}, \theta)$ 为离散 r.v. $\mathbf{X}$ 的概率分布时,式 (\ref{eq:Cramer_Rao_bound_general}) 变为
\begin{equation}
D_\theta[\hat{g}(\mathbf{X})] \ge \frac{[g'(\theta)]^2}{n \sum\limits_{i} \left\{ \left[\frac{\partial \log f(X_i, \theta)}{\partial \theta}\right]^2 f(x_i, \theta) \right\}}. \label{eq:Cramer_Rao_bound_discrete}
\end{equation}
\end{remark}

\begin{remark}
    若$\hat{g}(\mathbf{X})$达到 C-R 下界,则验证了该估计是UMVUE. 然而若达不到 C-R 下界,并不能说明 $g(\theta)$的UMVUE不存在。
\end{remark}

\begin{definition}[无偏估计的效率]\label{def:efficiency}
设 $\hat{g}(\mathbf{X})$ 为 $g(\theta)$ 的无偏估计,比值
\begin{equation}
e_{\hat{g}}(\theta) = \frac{(g'(\theta))^2 / (nI(\theta))}{D_\theta[\hat{g}(\mathbf{X})]} \label{eq:efficiency_ratio}
\end{equation}
称为无偏估计 $\hat{g}(\mathbf{X})$ 的\textbf{效率} (efficiency)。显然 $0 < e_g(\theta) \le 1$,当 $e_g(\theta) = 1$ 时,
称 $\hat{g}(\mathbf{X})$ 是 $g(\theta)$ 的\textbf{有效估计} (effective estimation)。若 $\hat{g}(\mathbf{X})$ 不是 $g(\theta)$ 的有效估
计,但 $\lim_{n\to\infty} e_g(\theta) = 1$,则称 $\hat{g}(\mathbf{X})$ 是 $g(\theta)$ 的\textbf{渐近有效估计} (asymptotically effective
estimation)。
\end{definition}

\section{区间估计}\label{sec:区间估计}
\subsection{基本概念}\label{subsec:基本概念}
\begin{definition}[区间估计]\label{def:interval_estimation}
设有一个参数分布族 $\mathcal{F}=\{f(\mathbf{x}, \theta), \theta \in \Theta\}$, $g(\theta)$ 是定义在参数
空间 $\Theta$ 上的一个已知函数,$\mathbf{X} = (X_1, \dots, X_n)$ 是从分布族某总体 $f(\mathbf{x}, \theta)$ 中抽取
的样本。令 $\hat{g}_1(\mathbf{X})$ 和 $\hat{g}_2(\mathbf{X})$ 为定义在样本空间 $\mathcal{X}$ 上,取值在 $\Theta$ 上的两个统计量,
且 $\hat{g}_1(\mathbf{X}) \leq \hat{g}_2(\mathbf{X})$,则称随机区间 $[\hat{g}_1(\mathbf{X}), \hat{g}_2(\mathbf{X})]$ 为 $g(\theta)$ 的一个区间估计 (interval
estimation)。
\end{definition}

\begin{definition}[置信水平与置信系数]\label{def:confidence_level_coefficient}
设随机区间 $[\hat{\theta}_1, \hat{\theta}_2]$ 为参数 $\theta$ 的一个区间估计,则 $[\hat{\theta}_1, \hat{\theta}_2]$ 包含 $\theta$
的概率 $P_\theta(\hat{\theta}_1 \le \theta \le \hat{\theta}_2)$ 称为此区间估计的\textbf{置信水平} (confidence level)。置信水平在
参数空间 $\Theta$ 上的下确界
$$\inf_{\theta \in \Theta} P_\theta(\hat{\theta}_1 \le \theta \le \hat{\theta}_2)$$
称为该区间估计的\textbf{置信系数} (confidence coefficient)。
\end{definition}
\begin{remark}
    精度和置信度相互制约,根据Neyman的提议,在保证置信系数达到指定要求的前提下尽可能提高精度。
\end{remark}

\begin{definition}[置信区间]\label{def:confidence_interval}
设 $[\hat{\theta}_1(\mathbf{X}), \hat{\theta}_2(\mathbf{X})]$ 是参数 $\theta$ 的一个区间估计,若对于给定的 $0 <
\alpha < 1$, 有
\begin{equation}
P_\theta(\hat{\theta}_1(\mathbf{X}) \le \theta \le \hat{\theta}_2(\mathbf{X})) \ge 1-\alpha, \quad \theta \in \Theta, \label{eq:confidence_interval_condition}
\end{equation}
则称 $[\hat{\theta}_1(\mathbf{X}), \hat{\theta}_2(\mathbf{X})]$ 是 $\theta$ 的\textbf{置信水平} (confidence level) 为 $1-\alpha$ 的\textbf{置信区间}
(confidence interval)。而 $\inf_{\theta \in \Theta} P_\theta(\hat{\theta}_1(\mathbf{X}) \le \theta \le \hat{\theta}_2(\mathbf{X}))$ 称为 $[\hat{\theta}_1(\mathbf{X}), \hat{\theta}_2(\mathbf{X})]$ 的置信系
数。
\end{definition}

\begin{definition}[单侧置信上限与置信下限]\label{def:one_sided_confidence_limit}
设 $\hat{\theta}_U(\mathbf{X})$ 和 $\hat{\theta}_L(\mathbf{X})$ 是定义在样本空间 $\mathcal{X}$ 上,在参数空间 $\Theta$ 上
取值的两个统计量。若对给定 $0 < \alpha < 1$, 有
\begin{align}
P_\theta(\theta \le \hat{\theta}_U(\mathbf{X})) &\ge 1-\alpha, \quad \text{一切 } \theta \in \Theta, \label{eq:upper_confidence_limit} \\
P_\theta(\hat{\theta}_L(\mathbf{X}) \le \theta) &\ge 1-\alpha, \quad \text{一切 } \theta \in \Theta, \label{eq:lower_confidence_limit}
\end{align}
则分别称 $\hat{\theta}_U(\mathbf{X})$ 和 $\hat{\theta}_L(\mathbf{X})$ 是 $\theta$ 的\textbf{置信水平为 $1-\alpha$ 的 (单侧) 置信上限} (upper
confidence limit ) 和\textbf{置信下限} (lower confidence limit)。上式左端概率在参数空间 $\Theta$
上的下确界分别称为置信上、下限的置信系数。

显然,对置信上限 $\hat{\theta}_U$ 而言,若 $E(\hat{\theta}_U)$ 越小,其精确度越高;对置信下限 $\hat{\theta}_L$ 而言,
若 $E(\hat{\theta}_L)$ 越大,则置信下限的精度越高。
\end{definition}

\subsection{枢轴变量法——正态总体}\label{subsec:枢轴变量法——正态总体}
\begin{proposition}[构造置信区间的一般步骤]\label{prop:构造置信区间的一般步骤}
构造置信区间的步骤如下:
\begin{enumerate}
    \item 找待估参数 $\mu$ 的一个良好点估计。
    \item 构造一个 $T(\mathbf{X})$ 和 $\mu$ 的函数 $\varphi(T, \mu)$, 使其满足
    \begin{enumerate}
        \item[(i)] 其表达式与待估参数 $\mu$ 有关,
        \item[(ii)] 其分布与待估参数 $\mu$ 无关。
    \end{enumerate}
    则称随机变量 $\varphi(T, \mu)$ 为枢轴变量。

    \item 对给定的 $0 < \alpha < 1$,决定两个常数 $a$ 和 $b$,使得
    \begin{equation}
    P_\mu(a \le \varphi(T, \mu) \le b) = 1-\alpha. \label{eq:pivotal_quantity_prob}
    \end{equation}
\end{enumerate}
解括号中的不等式得到 $\hat{\mu}_1(\mathbf{X}) \le \mu \le \hat{\mu}_2(\mathbf{X})$,则有
\begin{equation}
P_\mu(\hat{\mu}_1(\mathbf{X}) \le \mu \le \hat{\mu}_2(\mathbf{X})) = 1-\alpha. \label{eq:final_confidence_interval}
\end{equation}
这表明 $[\hat{\mu}_1(\mathbf{X}), \hat{\mu}_2(\mathbf{X})]$ 是 $\mu$ 的置信水平为 $1-\alpha$ 的置信区间。
\end{proposition}

\begin{proposition}[单个总体参数的置信区间]\label{prop:单个总体参数的置信区间}
    \begin{enumerate}
    \item $\sigma^2$ 已知,$\mu$ 的置信系数为$1-\alpha$的置信区间为
    \begin{equation}
        [\bar{X} - \frac{\sigma}{\sqrt{n}}u_{\alpha/2}, \bar{X} + \frac{\sigma}{\sqrt{n}}u_{\alpha/2}]
    \end{equation}
    \item $\sigma^2$ 未知,$\mu$ 的置信系数为$1-\alpha$的置信区间为
    \begin{equation}
        [\bar{X} - \frac{S}{\sqrt{n}}t_{n-1}(\alpha/2), \bar{X} + \frac{S}{\sqrt{n}}t_{n-1}(\alpha/2)]
    \end{equation}
    \item $\mu$ 已知,$\sigma^2$ 的置信系数为$1-\alpha$的置信区间为
    \begin{equation}
        [\frac{nS_\mu^2}{\chi_n^2(\alpha/2)},\frac{nS_\mu^2}{\chi_n^2(1-\alpha/2)}]
    \end{equation}
    \item $\mu$ 未知,$\sigma^2$ 的置信系数为$1-\alpha$的置信区间为
    \begin{equation}
        [\frac{(n-1)S^2}{\chi_{n-1}^2(\alpha/2)},\frac{(n-1)S^2}{\chi_{n-1}^2(1-\alpha/2)}]
    \end{equation}
\end{enumerate}
\end{proposition}

\begin{proposition}[均值差$b-a$的置信区间]\label{prop:均值差$b-a$的置信区间}
    \begin{enumerate}
    \item $m=n$ 时
    令 $Z_i = Y_i - X_i, i = 1,2, \dots, n$. 
    $b-a$ 的置信水平 $1-\alpha$ 的置信区间为 
    \begin{equation}
        [\bar{Z} - \frac{S_Z}{\sqrt{n}} t_{n-1}(\alpha/2), \bar{Z} + \frac{S_Z}{\sqrt{n}} t_{n-1}(\alpha/2)]
    \end{equation}

    \item $\sigma_1^2$ 和 $\sigma_2^2$ 已知
    $b-a$ 的置信水平 $1-\alpha$ 的置信区间为
    \begin{equation}
        [\bar{X}-\bar{Y} - \sqrt{\sigma_1^2/n + \sigma_2^2/m} u_{\alpha/2}, \bar{X}-\bar{Y} + \sqrt{\sigma_1^2/n + \sigma_2^2/m} u_{\alpha/2}]
    \end{equation}

    \item $\sigma_1^2 = \sigma_2^2 = \sigma^2$ 未知,令 $S_{\omega}^2 = \frac{nS_1^2 + mS_2^2}{n+m-2}$, 
    $b-a$ 的置信水平 $1-\alpha$ 的置信区间为
    \begin{equation}
        [\bar{Y}-\bar{X} - S_{\omega}\sqrt{1/n+1/m} t_{n+m-2}(\alpha/2), \bar{Y}-\bar{X} + S_{\omega}\sqrt{1/n+1/m} t_{n+m-2}(\alpha/2)]
    \end{equation}

    \item $\sigma_1^2 \ne \sigma_2^2$ 未知 (Behrens-Fisher 问题)
    \begin{enumerate}
        \item $m,n$ 均大时,
    由引理 \ref{lem:slutsky}知
    $\frac{(\bar{Y}-\bar{X})-(b-a)}{\sqrt{S_1^2/m + S_2^2/n}} \xrightarrow{\mathscr{L}} N(0,1)$.
    $b-a$ 的置信水平 $1-\alpha$ 的置信区间为
    \begin{equation}
        [\bar{X}-\bar{Y} - \sqrt{S_1^2/m + S_2^2/n} u_{\alpha/2}, \bar{X}-\bar{Y} + \sqrt{S_1^2/m + S_2^2/n} u_{\alpha/2}]
    \end{equation}
        \item 一般情况下的估计记不住,不写了。
    \end{enumerate}
\end{enumerate}
\end{proposition}

\begin{proposition}[方差比$\sigma_1^2/\sigma_2^2$的置信区间]\label{prop:方差比的置信区间}
    \begin{enumerate}
    \item $a, b$ 已知,$\frac{\sigma_1^2}{\sigma_2^2}$ 的置信水平 $1-\alpha$ 的置信区间为
    \begin{equation}
        [\frac{S_a^2}{S_b^2}\frac{1}{F_{m,n}(\alpha/2)}, \frac{S_a^2}{S_b^2}\frac{1}{F_{m,n}(1-\alpha/2)}]
    \end{equation}

    \item $a, b$ 未知,$\frac{\sigma_1^2}{\sigma_2^2}$ 的置信水平 $1-\alpha$ 的置信区间为
    \begin{equation}
        [\frac{S_1^2}{S_2^2}\frac{1}{F_{m-1,n-1}(\alpha/2)}, \frac{S_1^2}{S_2^2}\frac{1}{F_{n-1,m-1}(1-\alpha/2)}]
    \end{equation}
\end{enumerate}
\end{proposition}
\begin{remark}
    为查表方便,经常用命题 \ref{prop:f_distribution_properties} 的(3).
\end{remark}

\subsection{枢轴变量法——非正态总体}\label{subsec:枢轴变量法——非正态总体}
某些特殊的小样本情况下可以得到精确的枢轴变量分布。
\begin{proposition}[指数分布参数的置信区间]\label{prop:指数分布参数的置信区间}
设 $X_1, \dots, X_n$ 为从指数分布 $\operatorname{Exp}(\lambda)$ 中抽取的简单随机样本,其密度函数为
$f(x, \lambda) = \lambda e^{-\lambda x}I_{(0, \infty)}(x), \lambda > 0$, $\lambda$ 的置信系数为 $1-\alpha$ 的置信区间为
\begin{equation}
\left[\frac{\chi_{2n}^2(1-\alpha/2)}{2n\bar{X}}, \frac{\chi_{2n}^2(\alpha/2)}{2n\bar{X}}\right]. \label{eq:exp_lambda_ci}
\end{equation}
\end{proposition}
\begin{proof}
    因为 $\bar{X} = \frac{1}{n}\sum_{i=1}^n X_i$ 是 $1/\lambda$ 的一个无偏估计 (且是 UMVUE),设想 $\lambda$ 的置信区
间可通过 $\bar{X}$ 表示。枢轴变量可取为 $T = 2n\lambda\bar{X}$,由推论 \ref{cor:sum_of_exponentials_to_chi_squared} 可知
\begin{equation}
2n\lambda\bar{X} = 2\lambda \sum_{i=1}^n X_i \sim \chi_{2n}^2.\label{eq:exp_pivotal}
\end{equation}
确定 $a, b$ 使得
$$P(a \le 2n\lambda\bar{X} \le b) = 1-\alpha.$$
使上式成立的 $a, b$ 有很多对,其中存在一对,使置信区间的长度最短,但那对 $a, b$
表达式复杂,不易求得,应用上也不方便。通常采用下列方法,令
$$P(2n\lambda\bar{X} < a) = \frac{\alpha}{2}, \quad P(2n\lambda\bar{X} > b) = \frac{\alpha}{2}.$$
由式 (\ref{eq:exp_pivotal}) 可知,$a = \chi_{2n}^2(1-\alpha/2)$, $b = \chi_{2n}^2(\alpha/2)$,这样找到的 $a, b$ 虽不能使置信
区间的精度最高,但表达式简单,可通过查 $\chi^2$ 分布的上侧 $\alpha$ 分位数表求得,应用
上很方便。因此有
$$P\left(\chi_{2n}^2(1-\alpha/2) \le 2n\lambda\bar{X} \le \chi_{2n}^2(\alpha/2)\right) = 1-\alpha.$$
\end{proof}

\begin{proposition}[均匀分布参数的置信区间]\label{prop:均匀分布参数的置信区间}   
设 $\mathbf{X}=(X_1, \dots, X_n)$ 为自均匀分布总体 $U(0,\theta)$ 中抽取的简单随机样本,$\theta$
的置信系数为 $1-\alpha$ 的置信区间为
\begin{equation}
[d_1 T, d_2 T] = \left[T, \frac{T}{\sqrt[n]{\alpha}}\right]. \label{eq:uniform_theta_ci}
\end{equation}
\end{proposition}
\begin{proof}
    设 $T(\mathbf{X}) = X_{(n)} = \max\{X_1, \dots, X_n\}$,$T(\mathbf{X})$ 为充分统计量,$(n+1)T/n$ 是 $\theta$ 的无偏估计 (也是 UMVUE 的)。由于 $Y_i = X_i/\theta \sim U(0,1), i=1,2,\dots,n$,故 $Y_{(n)} = T/\theta$ 的密度函数为
$f(y) = ny^{n-1}I_{(0,1)}(y)$。取 $Z=1/Y_{(n)} = \theta/T$ 为枢轴变量,其表达式与 $\theta$ 有关,其
密度函数为
\begin{equation}
g(z) \sim n z^{-(n+1)}I_{(1,\infty)}(z). \label{eq:uniform_pivotal_density}
\end{equation}
$g(z)$ 与 $\theta$ 无关。确定 $d_1, d_2$, $1 \le d_1 < d_2 \le \infty$ 使得
\begin{align}
P_\theta\left(d_1 \le \frac{\theta}{T} \le d_2\right) &= P(d_1 T \le \theta \le d_2 T) \nonumber \\
&= \int_{d_1}^{d_2} n z^{-n-1}dz = -\frac{1}{n}z^{-n}\Big|_{d_1}^{d_2} \nonumber \\
&= \frac{1}{d_1^n} - \frac{1}{d_2^n} = 1-\alpha. \label{eq:uniform_prob_integral}
\end{align}
于是 $[d_1 T, d_2 T]$ 为 $\theta$ 的置信系数为 $1-\alpha$ 的置信区间。由式 (\ref{eq:uniform_prob_integral}) 中被积函数 $g(z)$
单调降序的性质,要使区间估计的精度最高,宜取 $d_1=1$,再解方程 $1-1/d_2^n = 1-\alpha$
得到 $d_2 = 1/\sqrt[n]{\alpha}$。
\end{proof}
大样本情况下主要用到Slutsky引理 \ref{lem:slutsky}.
\begin{proposition}[二项分布总体参数的置信区间]\label{prop:二项分布总体参数的置信区间}
    设 $\mathbf{X} = (X_1, \dots, X_n)$ 为自两点分布 $b(1,p)$ 中抽取的简单随机样本,$p$ 的置信系数近似为 $1-\alpha$ 的置信区间为
\begin{equation}
\left[\hat{p} - u_{\alpha/2}\sqrt{\hat{p}(1-\hat{p})/n}, \hat{p} + u_{\alpha/2}\sqrt{\hat{p}(1-\hat{p})/n}\right]. \label{eq:binomial_p_ci}
\end{equation}
这里$\hat{p} = S_n/n \stackrel{P}{\longrightarrow} p$.
\end{proposition}

\begin{proposition}[Poisson 分布参数的置信区间]\label{prop:Poisson 分布参数的置信区间}
    设 $\mathbf{X} = (X_1, \dots, X_n)$ 为抽自 Poisson 总体 $P(\lambda)$ 的简单随机样本,$\lambda$ 的置信系数近似为 $1-\alpha$ 的置信区间为
\begin{equation}
\left[\hat{\lambda} - u_{\alpha/2}\sqrt{\hat{\lambda}/n}, \hat{\lambda} + u_{\alpha/2}\sqrt{\hat{\lambda}/n}\right]. \label{eq:poisson_lambda_ci}
\end{equation}
\end{proposition}
\begin{proof}
    由 $\hat{\lambda} = \bar{X} \stackrel{P}{\longrightarrow} \lambda$ (当 $n \to \infty$ 时),因此当
$n \to \infty$ 时有
$$\frac{\sqrt{n}(\hat{\lambda}-\lambda)}{\sqrt{\lambda}}, \quad \frac{\sqrt{\lambda}}{\sqrt{\hat{\lambda}}} \stackrel{P}{\longrightarrow} 1.$$
故由引理 \ref{lem:slutsky} 可知
\begin{equation}
\frac{\sqrt{n}(\hat{\lambda}-\lambda)}{\sqrt{\hat{\lambda}}} = \frac{\sqrt{n}(\hat{\lambda}-\lambda)}{\sqrt{\lambda}} \cdot \frac{\sqrt{\lambda}}{\sqrt{\hat{\lambda}}} \stackrel{\mathscr{L}}{\longrightarrow} N(0,1). \label{eq:poisson_pivotal_quantity}
\end{equation}
令 $T = \sqrt{n}(\hat{\lambda}-\lambda)/\sqrt{\hat{\lambda}}$ 为枢轴变量,其极限分布与未知参数 $\lambda$ 无关。给定置信系
数 $1-\alpha$,则有
$$P\left(\left|\frac{\sqrt{n}(\hat{\lambda}-\lambda)}{\sqrt{\hat{\lambda}}}\right| \le u_{\alpha/2}\right) \approx 1-\alpha.$$
\end{proof}







