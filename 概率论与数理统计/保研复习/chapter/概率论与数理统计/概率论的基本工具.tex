\chapter{概率论的基本工具}\label{chap:概率论的基本工具}
\begin{introduction}
        \item 概率空间\quad 定义\ref{def:概率空间}
        \item 条件概率\quad 定义 \ref{def:conditional_probability}
        \item 乘法公式 \quad 式\ref{eq:multiplication_formula}
        \item 全概率公式\quad 命题\ref{prop:全概率公式}
        \item 贝叶斯公式\quad 命题 \ref{prop:bayes_formula}
        \item 事件独立性\quad 定义 \ref{def:statistical_independence}
        \item 随机变量\quad 定义\ref{def:random_variable}
        \item 分布函数\quad 定义 \ref{def:distribution_function}
        \item 概率分布\quad 定义\ref{def:概率分布}
        \item 分布密度函数\quad 定义 \ref{def:pdf}
        \item 边际分布函数\quad 定义 \ref{def:边际分布}
        \item 条件分布 \quad 定义\ref{def:条件分布(离散型)}
        \item 随机变量函数的分布 \quad  命题\ref{prop:随机变量函数的分布}
        \item 数学期望 \quad 定义\ref{def:mathematical_expectation_general}
        \item 佚名统计学家公式\quad 命题 \ref{prop:anonymous_statistician_formula}
        \item 方差\quad 定义 \ref{def:variance}
        \item 协方差 \quad 定义\ref{def:covariance}
        \item 相关系数 \quad 定义\ref{def:correlation_coefficient}
        \item 原点矩 \quad 定义\ref{def:kth_raw_moment}
        \item 条件数学期望 \quad 定义\ref{def:conditional_expectation}
        \item 重期望公式  \quad 命题\ref{prop:law_of_total_expectation}
        \item 特征函数 \quad 定义\ref{def:characteristic_function}
\end{introduction}
\section{事件与概率}\label{sec:事件与概率}
\begin{definition}[样本空间]\label{def:样本空间}
    随机试验的结果为样本点,它们的全体构成样本空间 $\boldsymbol{\Omega}$。
\end{definition}
\begin{definition}[$\sigma$代数]\label{def:sigma代数}
    $\mathcal{F}$是空间 $\boldsymbol{\Omega}$ 上的一个集类,且满足:
\begin{enumerate}
    \item[(i)] $\boldsymbol{\Omega}\in\mathcal{F}$;
    \item[(ii)] 若 $A\in\mathcal{F}$,则 $\overline{A}\in\mathcal{F}$;
    \item[(iii)] 若 $A_n\in\mathcal{F}$,$n=1,2,\ldots$,则 $\bigcup_{n=1}^\infty A_n\in\mathcal{F}$。
\end{enumerate}
则称$\mathcal{F}$为 \textbf{$\sigma$ 域},亦称 \textbf{$\sigma$ 代数}。
\end{definition}
\begin{definition}[事件] \label{def:event_field}
若 $\mathcal{F}$ 是由样本空间 $\boldsymbol{\Omega}$ 的一些子集构成的一个 $\sigma$ 域,则称它为\textbf{事件域},$\mathcal{F}$ 中的元素称为\textbf{事件},$\boldsymbol{\Omega}$ 称为\textbf{必然事件},$\emptyset$ 称为\textbf{不可能事件}。
\end{definition}
\begin{example}[一维Borel\ $\sigma$域]\label{ex:一维Borelsigma域}
    称由一切形为 $[a,b)$ 的有界左闭右开区间构成的集类所产生的 $\sigma$ 域为\textbf{一维Borel\  $\sigma$ 域},记之为 $\mathcal{B}_1$,称 $\mathcal{B}_1$ 中的集为\textbf{一维Borel点集}。

若 $x,y$ 为任意实数,由于
\[
\{x\} = \bigcap_{n=1}^\infty \left[x,x+\frac{1}{n}\right)
\]
\[
(x,y) = [x,y) - \{x\}
\]
\[
[x,y] = [x,y) + \{y\}
\]
\[
(x,y] = [x,y) + \{y\} - \{x\}
\]
因此,$\mathcal{B}_1$ 中包含一切开区间,闭区间,闭半区间,单个实数,可列个实数,以及由它们经可列次逆、并、交运算而得出的集合。
\end{example}

\begin{definition}[概率] \label{def:probability}
定义在事件域 $\mathcal{F}$ 上的一个集合函数 $P$ 称为\textbf{概率},如果它满足如下三个要求:
\begin{enumerate}
    \item[(i)] $P(A) \ge 0$,对一切 $A \in \mathcal{F}$;
    \item[(ii)] $P(\boldsymbol{\Omega}) = 1$;
    \item[(iii)] 若 $A_i \in \mathcal{F}$,$i=1,2,\ldots$ 且两两互不相容,则
    \begin{equation} \label{eq:countable_additivity}
    P\left(\sum_{i=1}^\infty A_i\right) = \sum_{i=1}^\infty P(A_i)
    \end{equation}
\end{enumerate}
性质 (i) 称为\textbf{非负性},性质 (ii) 称为\textbf{规范性},性质 (iii) 称为\textbf{可列可加性}或\textbf{完全可加性}。
\end{definition}

\begin{proposition}[加法公式]\label{prop:加法公式}
\begin{equation}
    P(A \cup B) = P(A) + P(B) - P(AB)\label{eq:union_prob}
\end{equation}
\end{proposition}
\begin{proof}
    因 $A \cup B = A \cup (B - AB)$,而且 $A \cap (B - AB) = \emptyset$,故 $P(A \cup B) = P(A) + P(B - AB)$,又 $AB \subset B$,于是得到
\[
P(A \cup B) = P(A) + P(B) - P(AB)
\]
\end{proof}
\begin{proposition}[一般加法公式]\label{prop:一般加法公式}
    若 $A_1,A_2,\ldots,A_n$ 为 $n$ 个事件,则
\begin{equation} \label{eq:general_addition_formula}
\begin{aligned}
P(A_1 \cup A_2 \cup \cdots \cup A_n) &= \sum_{i=1}^n P(A_i) - \sum_{\substack{i<j \\ i,j=1,\ldots,n}} P(A_i A_j) \\
&\quad + \sum_{\substack{i<j<k \\ i,j,k=1,\ldots,n}} P(A_i A_j A_k) - \cdots + (-1)^{n-1}P(A_1 A_2 \cdots A_n)
\end{aligned}
\end{equation}
\end{proposition}

\begin{definition}[概率空间]\label{def:概率空间}
    在科尔莫哥洛夫的概率论公理化结构中,称\textbf{三元总体}$(\boldsymbol{\Omega},\mathcal{F},P)$为\textbf{概率空间},其中 $\boldsymbol{\Omega}$ 是样本空间,$\mathcal{F}$ 是事件域,$P$ 是概率。
\end{definition}

\section{条件概率与统计独立性}\label{sec:条件概率与统计独立性}
\subsection{条件概率、全概率公式、Bayes公式}\label{subsec:条件概率、全概率公式、Bayes公式}
\begin{definition}[条件概率] \label{def:conditional_probability}
设 $(\boldsymbol{\Omega},\mathcal{F},P)$ 是一个概率空间,$B \in \mathcal{F}$,而且 $P(B)>0$,则对任意 $A \in \mathcal{F}$,记
\begin{equation} \label{eq:conditional_probability_formula}
P(A \mid B) = \frac{P(AB)}{P(B)}
\end{equation}
并称 $P(A \mid B)$ 为在事件 $B$ 发生的条件下事件 $A$ 发生的\textbf{条件概率}。
\end{definition}
\begin{remark}\label{rmk:乘法公式}
    由 \eqref{def:conditional_probability} 式立刻得到
\begin{equation} \label{eq:multiplication_formula}
P(AB) = P(B)P(A \mid B)
\end{equation}
这个等式被称为\textbf{概率的乘法公式}。
若还有 $P(A)>0$,则也可定义 $P(B \mid A)$,这时有
\begin{equation} \label{eq:multiplication_formula_symmetric}
P(AB) = P(A)P(B \mid A) = P(B)P(A \mid B)
\end{equation}
\end{remark}
\begin{proposition}[推广的乘法公式]\label{prop:推广的乘法公式}
    任意 $n$ 个事件之交的场合的乘法公式:
\begin{equation} \label{eq:generalized_multiplication_formula}
P(A_1 A_2 \cdots A_n) = P(A_1)P(A_2 \mid A_1)P(A_3 \mid A_1 A_2)\cdots P(A_n \mid A_1 A_2 \cdots A_{n-1})
\end{equation}
\end{proposition}
\begin{proposition}[全概率公式]\label{prop:全概率公式}
    \begin{equation} \label{eq:total_probability_formula}
P(B) = \sum_{i=1}^\infty P(A_i)P(B \mid A_i)
\end{equation}
\end{proposition}
\begin{proposition}[\textbf{贝叶斯公式}] \label{prop:bayes_formula}
若事件 $B$ 能且只能与两两互不相容的事件 $A_1,A_2,\ldots,A_n,\ldots$ 之一同时发生,即 $B = \sum_{i=1}^\infty BA_i$,且 $P(B) > 0$,则
\begin{equation} \label{eq:bayes_formula}
P(A_i \mid B) = \frac{P(A_i)P(B \mid A_i)}{\sum_{j=1}^\infty P(A_j)P(B \mid A_j)}
\end{equation}
\end{proposition}

\begin{proof}
根据条件概率的定义 \eqref{eq:conditional_probability_formula},我们有
\[
P(A_i \mid B) = \frac{P(A_i B)}{P(B)}
\]
再利用概率的乘法公式 \eqref{eq:multiplication_formula},分子可以写为 $P(A_i B) = P(A_i)P(B \mid A_i)$。

而分母 $P(B)$ 根据全概率公式 \eqref{eq:total_probability_formula} 可写为
\[
P(B) = \sum_{j=1}^\infty P(A_j)P(B \mid A_j)
\]
将这两部分代入条件概率的定义,即可得到贝叶斯公式:
\[
P(A_i \mid B) = \frac{P(A_i)P(B \mid A_i)}{\sum_{j=1}^\infty P(A_j)P(B \mid A_j)}
\]
定理证毕。
\end{proof}
\subsection{事件独立性}\label{subsec:事件独立性}
\begin{definition}[独立性] \label{def:statistical_independence}
对事件 $A$ 及 $B$,若
\begin{equation} \label{eq:statistical_independence_formula}
P(AB) = P(A)P(B)
\end{equation}
则称它们是\textbf{统计独立的},简称\textbf{独立的}。
\end{definition}
\begin{definition}[三个事件的独立性] \label{def:mutual_independence_three_events}
对于三个事件 $A,B,C$,若下列四个等式同时成立,则称它们\textbf{相互独立}。
\begin{gather}
P(AB)=P(A)P(B) \notag \\
P(BC)=P(B)P(C) \notag \\
P(AC)=P(A)P(C) \label{eq:pairwise_independence_three_events} \\
P(ABC)=P(A)P(B)P(C) \label{eq:mutual_independence_three_events}
\end{gather}
\end{definition}
\begin{example}[ Bernstein反例]
    一个均匀的正四面体,其第一面染成红色,第二面染成白色,第三面染成黑色,而第四面同时染上红,白,黑三种颜色。现在以 $A,B,C$ 分别记投一次四面体出现红,白,黑颜色朝下的事件,则由于在四面体中有两面有红色,因此
\[
P(A)=\frac{1}{2}
\]
同理 $P(B)=P(C)=\frac{1}{2}$,容易算出
\[
P(AB)=P(BC)=P(AC)=\frac{1}{4}
\]
所以 \eqref{eq:pairwise_independence_three_events} 成立,即 $A,B,C$ 两两独立,但是
\[
P(ABC) = \frac{1}{4} \ne \frac{1}{8} = P(A)P(B)P(C)
\]
因此 \eqref{eq:mutual_independence_three_events} 不成立,从而 $A,B,C$ 不相互独立。
\end{example}
\section{随机变量与分布函数}\label{sec:rvandcdf}
随机变量是样本点的一个函数。
\begin{definition}[随机变量] \label{def:random_variable}
设 $\xi(\omega)$ 是定义于概率空间 $(\boldsymbol{\Omega},\mathcal{F},P)$ 上的\textbf{单值实函数},如果对于直线上任一Borel点集 $B$,有
\begin{equation} \label{eq:measurable_function_condition}
\{\omega:\xi(\omega) \in B\} \in \mathcal{F}
\end{equation}
则称 $\xi(\omega)$ 为\textbf{随机变量},而 $P\{\xi(\omega) \in B\}$ 称为随机变量(r.v.)$\xi(\omega)$ 的\textbf{概率分布}。
\end{definition}
\begin{definition}[分布函数] \label{def:distribution_function}
称
\begin{equation} \label{eq:distribution_function_formula}
F(x) = P\{\xi(\omega) < x\},\quad -\infty < x < \infty
\end{equation}
为随机变量 $\xi(\omega)$ 的\textbf{分布函数} (distribution function)。
\end{definition}
\subsection{离散型随机变量}
\begin{definition}[概率分布]\label{def:概率分布}
    设 $\{x_i\}$ 为离散型随机变量 $\xi$ 的所有可能值; 而 $p(x_i)$ 是 $\xi$ 取 $x_i$ 的概率,即
\begin{equation} \label{eq:discrete_prob_mass_function}
P\{ \xi=x_i \} = p(x_i),\quad i=1,2,3,\ldots
\end{equation}
$\{p(x_i),i=1,2,3,\ldots \}$ 称为随机变量 $\xi$ 的\textbf{概率分布},它应满足下面关系:
\begin{equation} \label{eq:pmf_non_negative}
p(x_i) \ge 0,\quad i=1,2,3,\ldots
\end{equation}
\begin{equation} \label{eq:pmf_sum_to_one}
\sum_{i=1}^\infty p(x_i) = 1
\end{equation}
\end{definition}
\begin{remark}
    常见的离散型随机变量的概率分布见 \ref{sec:离散型}节。
\end{remark}
\subsection{连续型随机变量}\label{subsec:连续型随机变量}
\begin{definition}[分布密度函数]\label{def:pdf}
    连续型随机变量 $\xi$ 可取某个区间 $[c,d]$ 或 $(-\infty,\infty)$ 中的一切值,而且其分布函数 $F(x)$ 是\textbf{绝对连续函数},即存在可积函数 $p(x)$,使
\begin{equation} \label{eq:cdf_from_pdf}
F(x) = \int_{-\infty}^x p(y) dy
\end{equation}
称 $p(x)$ 为 $\xi$ 的 (\textbf{分布})\textbf{密度函数} (density function)。
\end{definition}
\begin{remark}
    显然
\begin{equation} \label{eq:pdf_from_cdf}
p(x) = F'(x)
\end{equation}
由分布函数的性质可知对 $p(x)$ 应有
\begin{equation} \label{eq:pdf_non_negative}
p(x)\ge 0
\end{equation}
\begin{equation} \label{eq:pdf_integral_one}
\int_{-\infty}^\infty p(x) dx = 1
\end{equation}
反之,对于定义在 $(-\infty,\infty)$ 上的可积函数 $p(x)$,若它满足 \eqref{eq:pdf_non_negative} 及 \eqref{eq:pdf_integral_one},则由 \eqref{eq:cdf_from_pdf} 定义的函数 $F(x)$ 是一个分布函数。
\end{remark}
\begin{remark}
    对任意实数 $c$,计算 $P\{ \xi=c \}$,因为
\[
P\{ \xi=c \} \le P\{ c \le \xi < c+h \} = \int_c^{c+h} p(x)dx
\]
故
\[
0 \le P\{ \xi=c \} \le \lim_{h\to 0} \int_c^{c+h} p(x)dx = 0
\]
因此
\[
P\{ \xi=c \} = 0
\]
\textbf{一个事件的概率等于零,这事件并不一定是不可能事件}; 同样地,一个事件的概率等于 $1$,这事件也不一定是必然事件。
\end{remark}

\section{多元随机变量}\label{sec:多元随机变量}
\subsection{边际分布}\label{subsec:边际分布}
\begin{definition}[n维随机变量] \label{def:n_dimensional_random_vector}
若随机变量 $\xi_1(\omega),\xi_2(\omega),\ldots,\xi_n(\omega)$ 定义在同一概率空间 $(\boldsymbol{\Omega},\mathcal{F},P)$ 上,则称
\begin{equation} \label{eq:n_dimensional_random_vector}
\boldsymbol{\xi}(\omega) = (\xi_1(\omega),\xi_2(\omega),\ldots,\xi_n(\omega))
\end{equation}
构成一个 $n$ \textbf{维随机向量},亦称 $n$ \textbf{维随机变量}。显然,一维随机向量即为随机变量。
\end{definition}

\begin{definition}[联合分布函数] \label{def:joint_distribution_function}
称 $n$ 元函数
\begin{equation} \label{eq:joint_distribution_function_formula}
F(x_1,x_2,\ldots,x_n) = P\{\xi_1(\omega)<x_1,\xi_2(\omega)<x_2,\ldots,\xi_n(\omega)<x_n\}
\end{equation}
为随机向量 $\boldsymbol{\xi}(\omega) = (\xi_1(\omega),\xi_2(\omega),\ldots,\xi_n(\omega))$ 的 (\textbf{联合}) \textbf{分布函数}。
\end{definition}
\begin{remark}
    给定了联合分布函数后,可以计算事件 $\{a_1 \le \xi_1 < b_1,a_2 \le \xi_2 < b_2,\ldots,a_n \le \xi_n < b_n\}$ 的概率,例如当 $n=2$ 时,
\begin{equation} \label{eq:prob_for_n_2}
P\{a_1 \le \xi_1 < b_1,a_2 \le \xi_2 < b_2\} = F(b_1,b_2)-F(a_1,b_2)-F(b_1,a_2)+F(a_1,a_2)
\end{equation}
\end{remark}
\begin{definition}[边际分布函数]\label{def:边际分布}
    一般地,若 $(\xi,\eta)$ 是二维随机向量,其分布函数为 $F(x,y)$,我们能由 $F(x,y)$ 得出 $\xi$ 或 $\eta$ 的分布函数。事实上,
\begin{equation} \label{eq:marginal_cdf_xi}
F_1(x) = P\{\xi < x\} = P\{\xi < x,\eta < +\infty \} = F(x,+ \infty)
\end{equation}
同理
\[
F_2(y) = P\{\eta < y\} = F(+\infty,y)
\]
$F_1(x)$ 及 $F_2(y)$ 称为 $F(x,y)$ 的\textbf{边际分布函数}。
\end{definition}

\begin{definition}[边际密度函数]\label{def:边际密度函数}
    若 $F(x,y)$ 是连续型分布函数,有密度函数 $p(x,y)$,那么
\[
F_1(x) = \int_{-\infty}^x \int_{-\infty}^\infty p(u,y) dudy
\]
因此 $F_1(x)$ 是连续型分布函数,其密度函数为
\begin{equation} \label{eq:marginal_pdf_x}
p_1(x) = \int_{-\infty}^\infty p(x,y) dy
\end{equation}
同理 $F_2(x)$ 是连续型分布函数,其密度函数为
\begin{equation} \label{eq:marginal_pdf_y}
p_2(y) = \int_{-\infty}^\infty p(x,y) dx
\end{equation}
$p_1(x)$ 及 $p_2(y)$ 称为 $p(x,y)$ 的\textbf{边际(分布)密度函数}。
\end{definition}
\subsection{条件分布}\label{subsec:条件分布}
对二维的场合的离散型随机变量
\begin{definition}[条件分布(离散型)]\label{def:条件分布(离散型)}
若已知 $\xi = x_i (p_1(x_i) > 0)$,则事件 $\{\eta = y_j\}$ 的条件概率为
\begin{equation} \label{eq:conditional_prob_discrete}
P\{ \eta = y_j \mid \xi = x_i \} = \frac{P\{ \xi = x_i,\eta = y_j \}}{P\{ \xi = x_i \}} = \frac{p(x_i,y_j)}{p_1(x_i)}
\end{equation}
该式定义了随机变量 $\eta$ 关于随机变量 $\xi$ 的\textbf{条件分布}。
\end{definition}
连续型则有
\begin{definition}[条件分布(连续型)]\label{def:条件分布(连续型)}
    因此在给定 $\xi = x$ 的条件下,$\eta$ 的分布密度函数为
\begin{equation} \label{eq:conditional_pdf_y_given_x}
p(y \mid x) = \frac{p(x,y)}{p_1(x)}
\end{equation}
在给定 $\eta = y$ 的条件下,$\xi$ 的分布密度函数为
\begin{equation} \label{eq:conditional_pdf_x_given_y}
p(x \mid y) = \frac{p(x,y)}{p_2(y)}
\end{equation}
\end{definition}
\section{随机变量的函数及其分布}\label{sec:随机变量的函数及其分布}
这里的一般问题是: 已知随机变量 $\xi$ 的分布函数 $F(x)$ 或密度函数 $p(x)$,要求 $\eta=g(\xi)$ 的分布函数 $G(y)$ 或密度函数 $q(y)$。
\begin{equation} \label{eq:transformed_cdf_integral}
G(y)=P\{\eta<y\}=P\{g(\xi)<y\} = \int_{g(x)<y} p(x) dx
\end{equation}
若 $\eta=g(\xi_1,\ldots,\xi_n)$,而 $(\xi_1,\ldots,\xi_n)$ 的密度函数为 $p(x_1,\ldots,x_n)$,则
\begin{equation} \label{eq:transformed_cdf_multivariate_integral}
G(y) = P\{\eta < y\} = \idotsint_{g(x_1,\ldots,x_n) < y} p(x_1,\ldots,x_n) dx_1 \cdots dx_n
\end{equation}

若$(\xi_1,\ldots,\xi_n)$的密度函数为 $p(x_1,\ldots,x_n)$,求 $\eta_1 = g_1(\xi_1,\ldots,\xi_n),\ldots,\eta_m = g_m(\xi_1,\ldots,\xi_n)$ 的分布。这时有
\begin{equation} \label{eq:joint_cdf_transformed}
G(y_1,\ldots,y_m) = P\{\eta_1<y_1,\ldots,\eta_m<y_m\} = \idotsint_{\substack{g_1(x_1,\ldots,x_n)<y_1 \\ \ldots \\ g_m(x_1,\ldots,x_n)<y_m}} p(x_1,\ldots,x_n) dx_1\cdots dx_n
\end{equation}

\begin{proposition}[随机变量函数的分布]\label{prop:随机变量函数的分布}
    如果对 $y_i=g_i(x_1,\ldots,x_n)$,$i=1,2,\ldots,n$,存在唯一的反函数 $x_i(y_1,\ldots,y_n)=x_i(i=1,\ldots,n)$,而且 $(\eta_1,\ldots,\eta_n)$ 的密度函数为 $q(y_1,\ldots,y_n)$,那么
\begin{equation} \label{eq:transformed_joint_cdf_integral}
G(y_1,\ldots,y_n) = \idotsint_{\substack{u_1<y_1 \\ \ldots \\ u_n<y_n}} q(u_1,\ldots,u_n) du_1\cdots du_n
\end{equation}比较 $m=n$ 时的 \eqref{eq:joint_cdf_transformed} 与 \eqref{eq:transformed_joint_cdf_integral} 可知
\begin{equation} \label{eq:transformed_joint_pdf}
q(y_1,\ldots,y_n) = \begin{cases} p(x_1(y_1,\ldots,y_n),\ldots,x_n(y_1,\ldots,y_n))|J|,& \text{若}(y_1,\ldots,y_n)\text{属于 }g_1,\ldots,g_n\text{的值域} \\ 0,& \text{其他} \end{cases}
\end{equation}
其中 $J$ 为坐标变换的雅可比行列式
\begin{equation} \label{eq:jacobian_determinant}
J=
\begin{vmatrix}
\frac{\partial x_1}{\partial y_1} & \cdots & \frac{\partial x_1}{\partial y_n} \\
\vdots & \ddots & \vdots \\
\frac{\partial x_n}{\partial y_1} & \cdots & \frac{\partial x_n}{\partial y_n}
\end{vmatrix}
\end{equation}
这里,我们假定上述偏导数存在而且连续。
\end{proposition}

\begin{example}[ 随机变量和的分布]\label{ex:随机变量和的分布}
 若 $\eta = \xi_1 + \xi_2$, 而 $(\xi_1, \xi_2)$ 的密度函数为 $p(x_1, x_2)$, 则
\begin{align*}
G(y) &= P\{\eta < y\} = \int_{x_1+x_2<y} p(x_1, x_2) dx_1 dx_2 \\
&= \int_{-\infty}^\infty \left[ \int_{-\infty}^{y-x_1} p(x_1, x_2) dx_2 \right] dx_1
\end{align*}
特别当 $\xi_1, \xi_2$ 相互独立时, 有 $p(x_1, x_2) = p_1(x_1)p_2(x_2)$, 这里 $p_1(x_1)$ 为 $\xi_1$ 的密度函数, $p_2(x_2)$ 为 $\xi_2$ 的密度函数。
\begin{align*}
G(y) &= \int_{-\infty}^\infty \left[ \int_{-\infty}^{y-x_1} p_1(x_1)p_2(x_2) dx_2 \right] dx_1 \\
&= \int_{-\infty}^\infty \left[ p_1(x_1) \int_{-\infty}^{y-x_1} p_2(x_2) dx_2 \right] dx_1
\end{align*}
因此 $\eta$ 的密度函数为
\begin{equation} \label{eq:convolution_formula_1}
q(y) = \int_{-\infty}^\infty p_1(u) p_2(y - u) du
\end{equation}
也可写为
\begin{equation} \label{eq:convolution_formula_2}
q(y) = \int_{-\infty}^\infty p_1(y - u) p_2(u) du
\end{equation}
\end{example}
\begin{remark}
    \eqref{eq:convolution_formula_1} 或 \eqref{eq:convolution_formula_2} 称为\textbf{卷积公式}。
\end{remark}

\begin{example}[ 随机变量商的分布]\label{ex:随机变量商的分布}
    若 $\eta = \frac{\xi_1}{\xi_2}$, 而 $(\xi_1, \xi_2)$ 的密度函数为 $p(x_1, x_2)$, 则
\begin{align} \label{eq:quotient_distribution_cdf}
G(x) &= P\{\eta < x\} = P\left\{\frac{\xi_1}{\xi_2} < x\right\} = \int_{x_1/x_2<x} p(x_1, x_2) dx_1 dx_2 \notag \\
&= \int_0^\infty \left[ \int_{-\infty}^{zx} p(y,z) dy \right] dz + \int_{-\infty}^0 \left[ \int_{zx}^\infty p(y,z) dy \right] dz
\end{align}
$\eta$ 的密度函数为
\begin{equation} \label{eq:quotient_distribution_pdf}
q(x) = \int_0^\infty p(zx,z)z dz - \int_{-\infty}^0 p(zx,z)z dz = \int_{-\infty}^\infty |z|p(zx,z) dz
\end{equation}
\end{example}

\section{数字特征与特征函数}\label{sec:数字特征与特征函数}
\subsection{数学期望}\label{subsec:数学期望}
\begin{definition}[离散型随机变量的数学期望] \label{def:mathematical_expectation_discrete}
设 $\xi$ 为一离散型随机变量,它取值 $x_1,x_2,x_3,\ldots$ 对应的概率为 $p_1,p_2,p_3,\ldots$。如果级数
\begin{equation} \label{eq:mathematical_expectation_series}
\sum_{i=1}^\infty x_i p_i
\end{equation}
绝对收敛,则把它称为 $\xi$ 的\textbf{数学期望} (mathematical expectation),简称\textbf{期望}、\textbf{期望值}或\textbf{均值} (mean),记作 $\E\xi$。

\end{definition}
\begin{remark}
    当 $\sum_{i=1}^\infty |x_i|p_i$ 发散时,则说 $\xi$ 的数学期望不存在。
\end{remark}

\begin{definition}[连续型随机变量的数学期望] \label{def:mathematical_expectation_continuous}
设 $\xi$ 为具有密度函数 $p(x)$ 的连续型随机变量,当积分 $\int_{-\infty}^\infty xp(x)dx$ 绝对收敛时,我们称它为 $\xi$ 的\textbf{数学期望} (或\textbf{均值}),记作 $\E\xi$,即
\begin{equation} \label{eq:mathematical_expectation_integral}
\E\xi = \int_{-\infty}^\infty xp(x) dx
\end{equation}
\end{definition}
\begin{definition}[一般情形下的数学期望] \label{def:mathematical_expectation_general}
若 $\xi$ 的分布函数为 $F(x)$,则定义
\begin{equation} \label{eq:mathematical_expectation_stieltjes_integral}
\E\xi = \int_{-\infty}^\infty xdF(x)
\end{equation}
为 $\xi$ 的\textbf{数学期望} (或\textbf{均值})。上述积分需要绝对收敛,否则数学期望不存在。
\end{definition}
\begin{remark}
    这里应用了Stieltjes积分。
\end{remark}
随机变量的函数 $\eta=g(\xi)$ 的数学期望的定义,这里 $\xi$ 是分布函数为 $F_\xi(x)$ 的随机变量,而 $g(x)$ 是一元Borel函数。
\begin{proposition}[\textbf{佚名统计学家公式}] \label{prop:anonymous_statistician_formula}
\begin{equation} \label{eq:expectation_of_function_of_rv}
\E g(\xi) = \int_{-\infty}^\infty g(x)dF_\xi(x)
\end{equation}
这里要求这个积分绝对收敛。
\end{proposition}
\begin{proposition}[\textbf{数学期望的性质}] \label{prop:expectation_properties}
\begin{itemize}
    \item [(1)] 若 $a\le\xi\le b$,则 $a \le \E\xi \le b$。特别地 $\E c=c$,这里 $a,b,c$ 是常数。
    \item [(2)] 线性性质: 对任意常数 $c_i,i=1,\ldots,n$ 及 $b$,有
\[
\E\left(\sum_{i=1}^n c_i \xi_i + b\right) = \sum_{i=1}^n c_i \E\xi_i + b
\]
\end{itemize}
\end{proposition}
\subsection{方差,相关系数和矩}\label{subsec:方差,相关系数和矩}
\begin{definition}[方差] \label{def:variance}
若 $\E (\xi-\E\xi)^2$ 存在,则称它为随机变量 $\xi$ 的\textbf{方差} (variance),并记为 $\D\xi$,而 $\sqrt{\D\xi}$ 称为\textbf{根方差}、\textbf{均方差}或更多地称为\textbf{标准差} (standard deviation)。
\end{definition}
\begin{proposition}[\textbf{方差的计算公式}] \label{prop:variance_calculation}
利用数学期望的线性性质,随机变量 ξ 的方差 $\D\xi$ 可由以下公式计算:
\begin{equation} \label{eq:variance_calculation_formula}
\D\xi = \E (\xi-\E\xi)^2 = \E[\xi^2 - 2\xi \cdot \E\xi + (\E\xi)^2] = \E\xi^2 - 2\E\xi \cdot \E\xi + (\E\xi)^2 = \E\xi^2 - (\E\xi)^2
\end{equation}
在计算中,这个公式甚至比定义式更常用。
\end{proposition}
\begin{proposition}[\textbf{方差的性质}] \label{prop:variance_properties}
\begin{itemize}
    \item [(1)]常数的方差为 $0$。即 $\D c = 0$。
    \item [(2)]$\D(\xi+c) = \D\xi$,这里 $c$ 是常数。
    \item [(3)]$\D(c\xi) = c^2\D\xi$,这里 $c$ 是常数。
\end{itemize}
\end{proposition}
\begin{proposition}[\textbf{方差的极值性质}] \label{prop:variance_minimal_property}
若 $c \ne \E\xi$,则 $\D\xi < \E(\xi-c)^2$。
\end{proposition}

\begin{proof}
因为
\begin{align} \label{eq:variance_minimal_proof}
\D\xi &= \E(\xi-\E\xi)^2 \\
&= \E((\xi-c) - (\E\xi-c))^2 \notag \\
&= \E(\xi-c)^2 - (\E\xi-c)^2 \label{eq:variance_minimal_proof_step}
\end{align}
由于 $(\E\xi-c)^2 \ge 0$,且当 $c \ne \E\xi$ 时 $(\E\xi-c)^2 > 0$,因此 $\D\xi < \E(\xi-c)^2$。
\end{proof}
\begin{remark}
    这个性质表明数学期望具有一个重要的极值性质:在 $\E(\xi-c)^2$ 中,当 $c=\E\xi$ 时达到极小; 这也说明在 $\D\xi$ 的定义中取 $c=\E\xi$ 的合理性。
\end{remark}
\begin{definition}[协方差] \label{def:covariance}
称
\begin{equation} \label{eq:covariance_definition}
\sigma_{ij}=\text{cov}(\xi_i,\xi_j) = \E[ (\xi_i-\E\xi_i) (\xi_j-\E\xi_j) ],\quad i,j=1,2,\ldots,n
\end{equation}
为 $\xi_i$ 与 $\xi_j$ 的\textbf{协方差} (covariance)。
\end{definition}
\begin{proposition}[\textbf{协方差的性质}] \label{prop:covariance_properties}
\begin{itemize}
    \item [(1)]\begin{equation} \label{eq:covariance_calculation}
\text{cov}(\xi_i,\xi_j)=\E\xi_i\xi_j-\E\xi_i \cdot \E\xi_j
\end{equation}
    \item [(2)]\begin{equation} \label{eq:variance_of_sum}
\D\left(\sum_{i=1}^n \xi_i\right) = \sum_{i=1}^n \D\xi_i + 2 \sum_{1\le i<j\le n} \text{cov}(\xi_i,\xi_j)
\end{equation}
    \item [(3)]
\end{itemize}特别地
\begin{equation} \label{eq:variance_of_sum_two_vars}
\D(\xi_i\pm\xi_j) = \D\xi_i+\D\xi_j\pm2\text{cov}(\xi_i,\xi_j)
\end{equation}
\end{proposition}

\begin{remark}
    方差是协方差的特例,显然 $\sigma_{ii}=\D\xi_i$。
\end{remark}

\begin{definition}[协方差矩阵]\label{def:协方差矩阵}
    矩阵
\begin{equation} \label{eq:covariance_matrix}
\boldsymbol{\Sigma} = \begin{pmatrix}
\sigma_{11} & \sigma_{12} & \cdots & \sigma_{1n} \\
\sigma_{21} & \sigma_{22} & \cdots & \sigma_{2n} \\
\vdots & \vdots & \ddots & \vdots \\
\sigma_{n1} & \sigma_{n2} & \cdots & \sigma_{nn}
\end{pmatrix}
\end{equation}
称为 $\boldsymbol{\xi}$ 的\textbf{协方差矩阵},简记作 $\D\boldsymbol{\xi}$。
\end{definition}
 对任何实数 $t_j(j=1,2,\ldots,n)$ 有
\[
\sum_{j,k} \sigma_{jk}t_j t_k = \E\left[ \left(\sum_{j=1}^n t_j(\xi_j-\E\xi_j)\right)^2 \right] \ge 0
\]
因此 $\boldsymbol{\Sigma}$ 是一个非负定矩阵,所以若 $\det \boldsymbol{\Sigma}$ 记 $\boldsymbol{\Sigma}$ 的行列式,则有 $\det \boldsymbol{\Sigma} \ge 0$。
\begin{definition}[相关系数] \label{def:correlation_coefficient}
称
\begin{equation} \label{eq:correlation_coefficient_formula}
\rho_{ij}=\frac{\text{cov}(\xi_i,\xi_j)}{\sqrt{\D\xi_i}\sqrt{\D\xi_j}}
\end{equation}
为 $\xi_i$ 与 $\xi_j$ 的\textbf{相关系数} (correlation coefficient),这里当然要求 $\D\xi_i$ 与 $\D\xi_j$ 不为零。
\end{definition}
\begin{theorem}[\textbf{柯西-施瓦茨 (Cauchy-Schwarz) 不等式}] \label{thm:cauchy_schwarz}
对任意随机变量 $\xi$ 与 $\eta$ 都有
\begin{equation} \label{eq:cauchy_schwarz_inequality}
|\E\xi\eta|^2\le\E\xi^2 \cdot \E\eta^2
\end{equation}
等式成立当且仅当
\begin{equation} \label{eq:cauchy_schwarz_equality_condition}
P\{\eta=t_0\xi\}=1
\end{equation}
这里 $t_0$ 是某一个常数。
\end{theorem}

\begin{definition}[相关性] \label{def:uncorrelated}
若随机变量 $\xi$ 与 $\eta$ 的相关系数 $\rho=0$,则我们称 $\xi$ 与 $\eta$ \textbf{不相关}。
\end{definition}

\begin{proposition}[\textbf{不相关性的等价条件}] \label{prop:uncorrelated_equivalences}
对随机变量 $\xi$ 与 $\eta$,下面事实是等价的:
\begin{enumerate}
\item[(i)] $\text{cov}(\xi,\eta) = 0$;
\item[(ii)] $\xi$ 与 $\eta$ 不相关;
\item[(iii)] $\E\xi\eta = \E\xi \E\eta$;
\item[(iv)] $\D(\xi+\eta) = \D\xi+\D\eta$。
\end{enumerate}
\end{proposition}
\begin{proposition}[\textbf{独立性与不相关性}] \label{prop:independence_implies_uncorrelation}
若 $\xi$ 与 $\eta$ 独立,则 $\xi$ 与 $\eta$ 不相关。
\end{proposition}

\begin{proof}
我们只对连续型随机变量给出证明。因为 $\xi$ 与 $\eta$ 独立,故其联合密度函数 $p(x,y)=p_1(x)p_2(y)$,因此
\begin{align*}
\text{cov}(\xi,\eta) &= \E[(\xi-\E\xi)(\eta-\E\eta)] \\
&= \int_{-\infty}^\infty \int_{-\infty}^\infty (x - \E\xi)(y - \E\eta)p(x,y) dx dy \\
&= \int_{-\infty}^\infty \int_{-\infty}^\infty (x - \E\xi)(y - \E\eta)p_1(x)p_2(y) dx dy \\
&= \int_{-\infty}^\infty (x - \E\xi)p_1(x) dx \cdot \int_{-\infty}^\infty (y - \E\eta)p_2(y) dy \\
&= \E(\xi-\E\xi) \cdot \E(\eta-\E\eta) \\
&= (\E\xi - \E\xi) \cdot (\E\eta - \E\eta) \\
&= 0 \cdot 0 \\
&= 0
\end{align*}
由于 $\text{cov}(\xi,\eta) = 0$,根据不相关性的等价条件 \ref{prop:uncorrelated_equivalences} (i) 和 (ii) 是等价的,所以 $\xi$ 与 $\eta$ 不相关。
\end{proof}

\begin{example}[\textbf{不相关但非独立的反例}] \label{ex:uncorrelated_not_independent}
\newline 设 $\theta$ 服从 $[0,2\pi]$ 均匀分布,$\xi=\cos\theta$,$\eta=\cos(\theta+a)$,这里 $a$ 是定数。我们有
\[
\E\xi = \frac{1}{2\pi}\int_0^{2\pi} \cos t \,dt = 0
\]
\[
\E\eta = \frac{1}{2\pi}\int_0^{2\pi} \cos(t+a) \,dt = 0
\]
\[
\E\xi^2 = \frac{1}{2\pi}\int_0^{2\pi} \cos^2 t \,dt = \frac{1}{2\pi}\int_0^{2\pi} \frac{1+\cos(2t)}{2} \,dt = \frac{1}{2\pi}\left[\frac{t}{2}+\frac{\sin(2t)}{4}\right]_0^{2\pi} = \frac{1}{2}
\]
\[
\E\eta^2 = \frac{1}{2\pi}\int_0^{2\pi} \cos^2(t+a) \,dt = \frac{1}{2}
\]
\[
\E\xi\eta = \frac{1}{2\pi}\int_0^{2\pi} \cos t \cos(t+a) \,dt = \frac{1}{2\pi}\int_0^{2\pi} \frac{1}{2}[\cos(2t+a) + \cos a] \,dt = \frac{1}{2\pi}\left[\frac{\sin(2t+a)}{4} + \frac{t \cos a}{2}\right]_0^{2\pi} = \frac{1}{2}\cos a
\]
因此
\[
\text{cov}(\xi,\eta) = \E\xi\eta - \E\xi\E\eta = \frac{1}{2}\cos a - 0 \cdot 0 = \frac{1}{2}\cos a
\]
相关系数为
\[
\rho = \frac{\text{cov}(\xi,\eta)}{\sqrt{\D\xi}\sqrt{\D\eta}} = \frac{\frac{1}{2}\cos a}{\sqrt{\E\xi^2 - (\E\xi)^2}\sqrt{\E\eta^2 - (\E\eta)^2}} = \frac{\frac{1}{2}\cos a}{\sqrt{\frac{1}{2}}\sqrt{\frac{1}{2}}} = \cos a
\]
当 $a=0$ 时,$\rho=1$,$\xi=\eta$,存在完全线性关系。
当 $a=\pi$ 时,$\rho=-1$,$\xi=-\eta$,存在完全线性关系。
但是,当 $a=\frac{\pi}{2}$ 或 $\frac{3\pi}{2}$ 时,$\rho=0$,这时 $\xi$ 与 $\eta$ 不相关。不过,这时却有 $\xi^2+\eta^2=\cos^2\theta + \cos^2(\theta+\frac{\pi}{2}) = \cos^2\theta + (-\sin\theta)^2 = \cos^2\theta + \sin^2\theta = 1$。这是一个确定性关系,表明 $\xi$ 与 $\eta$ 之间存在非线性函数关系,因此 $\xi$ 与 $\eta$ 不独立。
\end{example}
\begin{remark}
    事实上,相关系数只是线性联系程度的一个度量。
\end{remark}
\begin{remark}
    二元正态分布/二值随机变量的情况下,独立性和不相关性是等价的。
\end{remark}
\begin{definition}[原点矩] \label{def:kth_raw_moment}
对正整数 $k$, 称
\begin{equation} \label{eq:kth_raw_moment}
m_k = \E\xi^k
\end{equation}
为 $k$ \textbf{阶原点矩},数学期望是一阶原点矩。
由于 $|x^{k-1}|\le 1+|x|^k$, 因此若 $k$ 阶矩存在,则所有低阶矩都存在。
\end{definition}

\begin{definition}[中心矩] \label{def:kth_central_moment}
对正整数 $k$,称
\begin{equation} \label{eq:kth_central_moment}
c_k = \E(\xi - \E\xi)^k
\end{equation}
为 $k$ \textbf{阶中心矩},方差是 2 阶中心矩。
\end{definition}

\begin{definition}[条件数学期望] \label{def:conditional_expectation}
两个随机变量 $\xi$ 与 $\eta$ 的场合下,假定它们具有密度函数 $p(x,y)$, 并以 $p(y|x)$ 记已知 $\xi=x$ 的条件下,$\eta$ 的条件密度函数,以 $p_1(x)$ 记 $\xi$ 的密度函数. 在 $\xi=x$ 的条件下,$\eta$ 的\textbf{条件数学期望}定义为
\begin{equation} \label{eq:conditional_expectation}
\E\{\eta \mid \xi = x\} = \int_{-\infty}^\infty y p(y \mid x) dy
\end{equation}
\end{definition}

若 $\xi,\eta$ 是偏依的随机变量,我们要找 $\xi$ 与 $\eta$ 的函数关系,设这个关系是 $y=h(x)$。如果 $\E\eta^2$ 及 $\E[h(\xi)]^2$ 都存在,我们的目的是找函数 $h(x)$,使 $\eta$ 与 $h(\xi)$ “尽可能靠近”,这里的“靠近”需要一个标准,最常用的是\textbf{高斯斯最小二乘法} (least squares),这时要求使如下的均方误差达到最小:
\begin{equation} \label{eq:least_squares_min}
\E[\eta - h(\xi)]^2 = \min
\end{equation}
因为
\begin{equation} \label{eq:least_squares_expansion}
\E[\eta - h(\xi)]^2 = \int_{-\infty}^\infty \int_{-\infty}^\infty [y - h(x)]^2 p(x,y)dxdy = \int_{-\infty}^\infty p_1(x)\left\{\int_{-\infty}^\infty [y - h(x)]^2 p(y \mid x) dy\right\} dx
\end{equation}
由 命题 \ref{prop:variance_minimal_property} 知,当 $h(x)=\E\{\eta \mid \xi=x\}$ 时,$\int_{-\infty}^\infty [y-h(x)]^2 \cdot p(y \mid x)dy$ 达到最小,从而使 \eqref{eq:least_squares_expansion} 达到最小,即当我们观察到 $\xi=x$ 时, $\E\{\eta \mid \xi=x\}$ 是对一切对 $\eta$ 的估计中均方误差最小的一个。

\begin{definition}[回归]\label{def:回归}
    称 $Y=\E\{\eta \mid \xi=x\}$ 是 $\eta$ 关于 $\xi$ 的回归。
\end{definition}
\begin{proposition}[\textbf{重期望公式}] \label{prop:law_of_total_expectation}
\begin{equation} \label{eq:law_of_total_expectation_formula}
\E\eta = \E[\E\{\eta \mid \xi\}]
\end{equation}
这是条件数学期望的一个重要的性质,称为\textbf{重期望公式}。
\end{proposition}

\begin{proof}
下面对连续型随机变量的场合加以证明。
\begin{align*}
\E[\E\{\eta \mid \xi\}] &= \int_{-\infty}^\infty \E\{\eta \mid \xi=x\}p_1(x) dx \\
&= \int_{-\infty}^\infty \left[ \int_{-\infty}^\infty yp(y \mid x) dy \right] p_1(x) dx \\
&= \int_{-\infty}^\infty \int_{-\infty}^\infty yp(y \mid x)p_1(x) dx dy \\
&= \int_{-\infty}^\infty \int_{-\infty}^\infty yp(x,y) dx dy \\
&= \E\eta
\end{align*}
命题得证。
\end{proof}
\subsection{特征函数}\label{subsec:特征函数}
\begin{definition}[特征函数] \label{def:characteristic_function}
若随机变量 $\xi$ 的分布函数为 $F_\xi(x)$,则称
\begin{equation} \label{eq:characteristic_function_formula}
f_\xi(t) = \E e^{it\xi} = \int_{-\infty}^\infty e^{itx} dF_\xi(x)
\end{equation}
为 $\xi$ 的\textbf{特征函数} (characteristic function)。
\end{definition}
\begin{remark}
    分布函数由特征函数唯一确定。
\end{remark}