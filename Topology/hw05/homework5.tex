\documentclass[12pt, a4paper, oneside]{ctexart}
\usepackage{amsmath, amsthm, amssymb, bm, color, framed, graphicx, hyperref, mathrsfs, mathtools, enumerate, tikz}
\usetikzlibrary{patterns}

\title{\textbf{Homework 5}}
\author{萃英学院\qquad 2022级\qquad 王一鑫}
\date{\today}
\linespread{1.5}
\newcounter{problemname}
\newenvironment{problem}{\begin{framed}\stepcounter{problemname}\par\noindent\textsc{Problem \arabic{problemname}. }}{\end{framed}\par}
\newenvironment{solution}{%
	\par\noindent\textsc{Solution. }\ignorespaces
}{%
	\hfill$\qed$\par
}
\newenvironment{note}{\par\noindent\textsc{Note of Problem \arabic{problemname}. }}{\\\par}

\begin{document}
	
	\maketitle
	
	\begin{problem}
		Let \( A \subseteq \mathbb{R} \) be the set of integers, and let \( X \) be the quotient space 
		\( \mathbb{R}/A \) obtained by collapsing \( A \) to a point as in Example 3.52. 
		(We are not using the notation \( \mathbb{R}/\mathbb{Z} \) for this space because that has a different meaning, described in Example 3.92.)

		\begin{enumerate}[(a)]
            
            \item Show that \( X \) is homeomorphic to a wedge sum of countably infinitely many circles. [Hint: express both spaces as quotients of a disjoint union of intervals.]

            \item Show that the equivalence class \( A \) does not have a countable neighborhood-basis in \( X \), so \( X \) is not first or second countable.
        \end{enumerate}
	
	\end{problem}
	
	\begin{solution}
			
		\begin{enumerate}[(a)]
			\item The wedge sum of countably infinitely many circles is the space
			\( B = \bigvee_{n\in A} S^1_n = \coprod_{n\in A} S^1_n / P \), 
			where $P$ is the set of distinguished points.
			Since \( S^1 \) is the quotient space 
			\( I / (0 \sim 1) \), we can express \( B \) as \( B = \coprod_{n\in A} I_n / \sim \), 
			where \( \sim \) is the relation \( p_i \sim p_j \) for all \( i, j \) and \( 0_i \sim 1_i \) 
			for all \( i \). With this one defines the quotient map \( q : \coprod_{n\in A} I_n \to B \) 
			by sending elements to their equivalence classes.

			Let \( \mathbb{R} = \coprod_{n\in A} I_n / (1_n \sim 0_{n+1}) \). 
			Then we find a quotient map \( r : \prod_{n\in A} I_n \to \mathbb{R} / A \) 
			that makes the same identifications as \( q \) to prove that the two spaces are 
			homeomorphic by the uniqueness of the quotient space.

			\item Let $X = \mathbb{R}\setminus A$ be the quotient space obtained by 
			collapsing the set of integers $A\subseteq\mathbb{R}$ to a single point. 
			
			The space $X$ is homeomorphic to a wedge sum of countably infinitely many circles 
			$\bigvee_{n\in A} S^1_n$, where each circle corresponds to the interval 
			$[n,n+1]$ with its endpoints identified to the common basepoint $A$.

			In the quotient topology, a neighborhood of \( A \) in \( X \) 
			corresponds to an open set in \( \mathbb{R} \) containing all integers 
			\( n \in \mathbb{Z} \), with each \( n \) surrounded by an open interval 
			\( (n - \varepsilon_n, n + \varepsilon_n) \) for some \( \varepsilon_n > 0 \). 
			These intervals are "collapsed" to form loops around \( A \) in \( X \).
			
			Suppose for contradiction that \( \{U_k\}_{k \in \mathbb{N}} \) is a countable neighborhood basis at \( A \).
			For each \( U_k \), there exists a corresponding open set in \( \mathbb{R} \) containing all integers 
			\( n \in \mathbb{Z} \), with intervals \( \left(n - \varepsilon_n^{(k)}, n + \varepsilon_n^{(k)}\right) \) 
			for some \( \varepsilon_n^{(k)} > 0 \).

			For each integer \( n \in \mathbb{Z} \), define \( \varepsilon = \frac{1}{2} \varepsilon_n^{(n)} \). 
			Construct an open set \( V \subseteq \mathbb{R} \) by taking:
			\[
			V = \bigcup_{n \in \mathbb{Z}} (n - \varepsilon, n + \varepsilon) .
			\]
			This \( V \) maps to a neighborhood \( U \) of \( A \) in \( X \).

			By construction, \( U \) does not contain any \( U_k \), contradicting the assumption that \( \{U_k\} \) is a neighborhood basis.

		\end{enumerate}

	\end{solution}
	
	\begin{problem}
		
		Let \( G \) be a topological group and let \( H \subseteq G \) be a subgroup. 
        Show that its closure \( \overline{H} \) is also a subgroup.
	
	
	\end{problem}
	
	\begin{solution}

	By hypothesis, let the product with inversion
	\[
	f : G \times G \to G \quad \text{defined by} \quad f(x, y) = xy^{-1}
	\]
	is continuous.

	Therefore, \( f^{-1}(\overline{H}) \) is closed. Now, notice that \( H \times H \subseteq f^{-1}(\overline{H}) \). So, taking closures,
	\[
	\overline{H \times H} \subseteq f^{-1}(\overline{H}).
	\]

	Now, we just have to show that \( \overline{H} \times \overline{H} \subseteq \overline{H \times H }\), to conclude that
	\[
	f(\overline{H} \times \overline{H}) \subseteq \overline{H}.
	\]
	In fact, For any topological spaces \( X, Y \) and subsets \( A \subseteq X \), \( B \subseteq Y \):
	\[
	\overline{A} \times \overline{B} \subseteq \overline{A \times B}
	\]Let \((x, y) \in \overline{A} \times \overline{B}\),
	\( W \) be any open neighborhood of \((x, y)\) in \( X \times Y \). 
	By the definition of product topology, there exists a basic open set 
	\( U_0 \times V_0 \subseteq W \) where
	\( U_0 \) is open in \( X \) containing \( x \) and
	\( V_0 \) is open in \( Y \) containing \( y \)

	From the closure properties:
	\begin{align*}
		U_0 \cap A &\neq \emptyset \quad (\text{since } x \in \overline{A}) \\
		V_0 \cap B &\neq \emptyset \quad (\text{since } y \in \overline{B})
	\end{align*}

	Therefore:
	\[
	(U_0 \times V_0) \cap (A \times B) = (U_0 \cap A) \times (V_0 \cap B) \neq \emptyset
	\]
	This implies \( W \cap (A \times B) \neq \emptyset \). Since \( W \) was an arbitrary neighborhood of \((x, y)\), it follows that \((x, y) \in \overline{A \times B}\).

	Take $A = B = H$, we obtain the result.


	\end{solution}
	
	\begin{problem}
		
		Suppose \( \Gamma \) is a normal subgroup of the topological group \( G \). Show that the quotient group \( G / \Gamma \) is a topological group with the quotient topology. 
        [Hint: it might be helpful to use Problems 3-5 and 3-22.]


	\end{problem}
	
	\begin{solution}
	Since $G$ is a topological group, then the usual inversion and 
	group multiplication operations on \( G \),
	\[
	i : G \to G \quad \text{given by} \quad i(g) = g^{-1}
	\]
	and
	\[
	m : G \times G \to G \quad \text{given by} \quad m(g, h) = gh,
	\]
	are continuous. We know from elementary abstract algebra that since
	\( \Gamma \) is normal, the quotient group \( G/\Gamma \) is itself a group 
	under the operations
	\[
	\bar{i} : G/\Gamma \to G/\Gamma \quad \text{given by} \quad \bar{i}(g\Gamma) = g^{-1}\Gamma
	\]
	and
	\[
	\bar{m} : G/\Gamma \times G/\Gamma \to G/\Gamma \quad \text{given by} \quad \bar{m}(g\Gamma, h\Gamma) = (gh)\Gamma.
	\]
	To show that \( G/\Gamma \) is a topological group, 
	it remains to show that these operations are continuous 
	with respect to the quotient topology on \( G/\Gamma \). 
	Now, let \( \pi : G \to G/\Gamma \) be the standard 
	projection map \( \pi(g) = g\Gamma \) for \( g \in G \).
	
	First, the \textbf{inversion map}. Let
	\[
	\psi = \pi \circ i : G \to G/\Gamma \quad \text{be given by} \quad \psi(g) = g^{-1}\Gamma
	\]
	which is clearly continuous as the composition of two continuous maps. 
	Then it's easy to see that \( \psi \) is constant on the fibers of \( \pi \):
	\[
	\pi(x) = \pi(y) \quad \Longleftrightarrow \quad x\Gamma = y\Gamma \quad \Longleftrightarrow \quad x = \gamma y \text{ for some } \gamma \in \Gamma
	\]
	hence
	\[
	\psi(x) = x^{-1} \Gamma = (\gamma y)^{-1} \Gamma = y^{-1} \gamma^{-1} \Gamma = y^{-1} \Gamma = \psi(y).
	\]

	Thus, \( \psi \) passes to the quotient to give rise to a (unique) 
	continuous map \( \overline{\psi} : G/\Gamma \to G/\Gamma \) 
	such that \( \overline{\psi} \circ \pi = \pi \circ i = \psi \); 
	it is obvious that \( \bar{\psi} = \bar{i} \), the inversion map of the group \( G/\Gamma \), 
	so inversion is continuous.

	Next, the \textbf{multiplication map}. Let
	\[
	\varphi = \pi \circ m : G \times G \to G/\Gamma
	\]
	which is continuous as the composition of continuous maps. 
	Define \( \pi \times \pi : G \times G \to G/\Gamma \times G/\Gamma \) 
	as the product of "two copies" of the quotient map on \( G \).
	
	First we will show that the map 
	\( p = \pi \times \pi : G \times G \to (G/\Gamma) \times (G/\Gamma) \) 
	is a quotient map. It is obviously surjective and continuous, 
	as the product of two surjective and continuous maps. 
	The quotient map \( \pi : G \to G/\Gamma \) is open (from Problem 3-22(a))
	and so \( p \) is open as the finite product of open maps 
	(from Problem 3-5), and so \( p \) is a quotient map.

	Then \( \varphi \) is constant on the fibers of \( \pi \times \pi \):
	\[
	\pi \times \pi (g, h) = \pi \times \pi (g', h') \quad \Longrightarrow \quad g = \gamma g', \, h = h' \gamma \text{ for some } \gamma, \gamma' \in \Gamma.
	\]
	Then
	\[
	\psi(g, h) = gh\Gamma = \gamma g' h' \gamma \Gamma = g'h'\Gamma = \psi(g', h')
	\]
	Since \( \Gamma \) is normal. Thus, \( \varphi \) passes to the quotient 
	and induces a (unique) continuous map \( \overline{\varphi} : G/\Gamma \times G/\Gamma \to G/\Gamma \) 
	such that \( \overline{\varphi} \circ (\pi \times \pi) = \pi \circ m = \varphi \). 
	Then clearly \( \overline{\psi} = \overline{m} \), the multiplication map 
	of the group \( G/\Gamma \), and so this multiplication is continuous.

	Thus, \( G/\Gamma \) is a topological group since both inversion and the 
	group multiplication operation are continuous (with respect to the quotient topology).
	
	\end{solution}
	
	
	
	\begin{problem}
		
        Consider the action of \( O(n) \) on \( \mathbb{R}^n \) 
		by matrix multiplication as in Example 3.88(b). 
		Prove that the quotient space is homeomorphic to \( [0, \infty) \). 
		[Hint: consider the function \( f : \mathbb{R}^n \to [0, \infty) \) 
		given by \( f(x) = |x| \).]


	\end{problem}
	
	\begin{solution}
	Recall that we have the action given as follows
	\[
	a : O(n) \times \mathbb{R}^n \to \mathbb{R}^n; \quad (M, x) \mapsto Mx
	\]
	But this map can be rewritten with
	\[
	a_M : \mathbb{R}^n \to \mathbb{R}^n; \quad x \mapsto Mx
	\]
	Then with this action we can define the orbits as follows:
	\[
	O(n) \cdot x = \{ a_M(x) : x \in \mathbb{R}^n \}
	\]
	on which we have the equivalence relation
	\[
	y R x \iff y \in a_M(x)
	\]
	Now we can define the quotient space as follows:
	\[
	Q := \mathbb{R}^n / O(n) = \{ [x] : x \in \mathbb{R}^n \}
	\]

	Now let us use the following function:
	\[
	\varphi : Q \to \mathbb{R}_{\geq 0}; \quad [x] \mapsto \|x\|
	\]
	I claim that this function is a homeomorphism. 
	To do so let us check injectivity. 
	Let \( [x], [y] \in Q \) s.t. \( \varphi([x]) = \varphi([y]) \iff \|x\| = \|y\| \). 
	This is equivalent to say \( \|Mx\| = \|My\| \) since \( M \) is orthogonal, 
	from where follows that \( [x] = [y] \). 
	For surjectivity, let \( r \in [0, \infty) \). 
	Then take \( R = [(r, 0_2, 0_3, \dots, 0_n)] \), then \( \varphi(R) = r \). 
	This shows that \( \varphi \) is bijective. 
	Furthermore, this function is clearly continuous since it is polynomial 
	in each component. To finish this proof, I claim that the inverse of 
	\( \varphi \) is given by
	\[
	\varphi^{-1} = \pi : \mathbb{R}_{\geq 0} \to Q; \quad x \mapsto [x]
	\]
	Indeed
	\[
	\varphi^{-1}(\varphi([x])) = \varphi^{-1}(\|x\|) = [x]
	\]
	and
	\[
	\varphi(\varphi^{-1}(x)) = \varphi([x]) = \|x\| = x
	\]
	But since we have endowed \( Q \) with the quotient topology 
	we know that \( \pi \) is continuous. Thus we have shown that it is homeomorphic.

		


	\end{solution}
	
	\begin{problem}
		
		Suppose \( X \) is a connected topological space, and \( \sim \) is an equivalence relation on 
        \( X \) such that every equivalence class is open. Show that there is exactly one equivalence class, namely \( X \) itself.

	\end{problem}
	
	\begin{solution}
		
		For each \( x \in X \), let \( U_x = \{ y \in X \mid x \sim y \} \). Let \( x_0 \in X \) be given. If \( U_{x_0} = X \), we are done. Suppose otherwise. Let 
	\[
	V = \bigcup_{x \in U_{x_0}^c} U_x.
	\]
	Then \( U_{x_0} \cup V = X \) and \( U_{x_0} \cap V = \emptyset \). \( U_{x_0} \) is open since it is an equivalence class. \( V \) is open since it is a union of equivalence classes, each of which is open.

	Therefore, \( X \) is disconnected. This is a contradiction, so \( U_{x_0} = X \).

	\end{solution}

\end{document}


