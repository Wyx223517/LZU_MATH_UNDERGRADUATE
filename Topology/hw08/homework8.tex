\documentclass[12pt, a4paper, oneside]{ctexart}
\usepackage{amsmath, amsthm, amssymb, bm, color, framed, graphicx, hyperref, mathrsfs, mathtools, enumerate, tikz}
\usepackage{float}
\usepackage{tikz-cd}
\usepackage{subcaption} 



\usetikzlibrary{patterns}

\title{\textbf{Homework 8}}
\author{萃英学院\qquad 2022级\qquad 王一鑫}
\date{\today}
\linespread{1.5}
\newcounter{problemname}
\newenvironment{problem}{\begin{framed}\stepcounter{problemname}\par\noindent\textsc{Problem \arabic{problemname}. }}{\end{framed}\par}
\newenvironment{solution}{%
	\par\noindent\textsc{Solution. }\ignorespaces
}{%
	\hfill$\qed$\par
}
\newenvironment{note}{\par\noindent\textsc{Note of Problem \arabic{problemname}. }}{\\\par}

\begin{document}
	
	\maketitle
	
	\begin{problem}
		
        For each of the following surface presentations, compute the Euler characteristic and determine which of our standard surfaces it represents.

        \begin{itemize}
          \item[(a)] $\langle a, b, c \mid abacb^{-1}c^{-1} \rangle$
          \item[(b)] $\langle a, b, c \mid abca^{-1}b^{-1}c^{-1} \rangle$
        \end{itemize}
        
	\end{problem}
    
	\begin{solution}
        
                \begin{itemize}
                  \item[(a)] \( \langle a, b, c \mid abacb^{-1}c^{-1} \rangle \)
                
                  The relator represents the boundary of a polygon with 6 edges. Each edge label appears once in forward and once in inverse form, indicating that we are gluing pairs of edges. So:
                
                  \[
                  F = 1, \quad E = \frac{6}{2} = 3, \quad V = 1
                  \]
                  \[
                  \chi = V - E + F = 1 - 3 + 1 = -1
                  \]
                
                  Since the Euler characteristic is odd, this must be a non-orientable surface. For non-orientable surfaces:
                
                  \[
                  \chi = 2 - k \Rightarrow k = 3
                  \]
                
                  Thus, the surface is the connected sum of 3 real projective planes $\mathbb{P}^2\#\mathbb{P}^2\#\mathbb{P}^2$.
                
                  \item[(b)] \( \langle a, b, c \mid abca^{-1}b^{-1}c^{-1} \rangle \)
                
                  This relator also gives a 6-gon with edge pairings. The calculation is the same:
                
                  \[
                  F = 1, \quad E = 3, \quad V = 2 \Rightarrow \chi = 0
                  \]
                
                  An orientable surface with $\chi = 0$ corresponds to the torus $\mathbb{T}^2$(orientable genus 1).                
                \end{itemize}   

		
	\end{solution}
		
	\begin{problem}
		
        Let $X$ be a topological space. Show that for any points $p, q \in X$, path 
        homotopy is an equivalence relation on the set of all paths in $X$ 
        from $p$ to $q$.

	\end{problem}
	
	\begin{solution}
		
                Let \( X \) be a topological space and let \( p, q \in X \). Consider the set of all paths from \( p \) to \( q \), i.e., continuous maps \( f: [0,1] \to X \) with \( f(0) = p \), \( f(1) = q \).

                We define an equivalence relation called path homotopy. Two paths \( f, g \) from \( p \) to \( q \) are path homotopic, denoted \( f \simeq g \), if there exists a continuous map \( H: [0,1] \times [0,1] \to X \) such that:
                \[
                H(s, 0) = f(s),\quad H(s, 1) = g(s),\quad H(0, t) = p,\quad H(1, t) = q
                \]
                for all \( s, t \in [0,1] \).

                We verify that this defines an equivalence relation.

                \textbf{(1) Reflexivity:}  
                Let \( f \) be a path from \( p \) to \( q \). Define \( H(s, t) = f(s) \). Then \( H \) is constant in \( t \) and clearly satisfies the conditions above, hence \( f \simeq f \).

                \textbf{(2) Symmetry:}  
                If \( f \simeq g \) via \( H \), define \( K(s, t) = H(s, 1 - t) \). Then \( K \) is continuous and satisfies:
                \[
                K(s, 0) = H(s, 1) = g(s),\quad K(s, 1) = H(s, 0) = f(s)
                \]
                with the same endpoints \( K(0, t) = p \), \( K(1, t) = q \). Hence \( g \simeq f \).

                \textbf{(3) Transitivity:}  
                Suppose \( f \simeq g \) via \( H \) and \( g \simeq h \) via \( G \). Define
                \[
                F(s, t) = 
                \begin{cases}
                H(s, 2t), & 0 \le t \le \tfrac{1}{2} \\
                G(s, 2t - 1), & \tfrac{1}{2} \le t \le 1
                \end{cases}
                \]
                This function is continuous by the pasting lemma and gives a homotopy from \( f \) to \( h \), so \( f \simeq h \).
		
	\end{solution}
	
	\begin{problem}
        
        Let $X$ be a path-connected topological space.
        \begin{itemize}
        \item[(a)] Let $f, g : I \to X$ be two paths from $p$ to $q$. Show that $f \sim g$ if and only if $f \cdot \overline{g} \sim c_p$.
        \item[(b)] Show that $X$ is simply connected if and only if any two paths in $X$ with the same initial and terminal points are path-homotopic.
        \end{itemize}

	\end{problem}
	
	\begin{solution}
               
                \begin{enumerate}[(a)]
                        \item   ($\Rightarrow$) 
                                Suppose \( f \sim g \). I'll show \( f \sim g \Rightarrow f \overline{g} \sim c_p \). 
                                There is a homotopy $H(s,t)$ satisfies 
                                \[
                                H(s,0) = f(s)\quad H(s,1) = g(s)\quad H(0,t) = p\quad H(1,t) = q
                                \]
                                Explicitly, we define our new homotopy as
                                \[
                                K(s, t) =
                                \begin{cases}
                                f(2s) & \text{for } s \in [0, \frac{1-t}{2}] \\
                                H(1 - t, \frac{1}{t}(s - \frac{1 - t}{2})) & \text{for } s \in [\frac{1 - t}{2}, \frac{1 + t}{2}] \\
                                g(2 - 2s) & \text{for } s \in [\frac{1 + t}{2}, 1]
                                \end{cases}
                                \]
                                Then 
                                \[
                                K(s,0) = f\cdot \bar{g}(s)\quad K(s,1) = p\quad K(0,t) = p\quad K(1,t) = p
                                \]
                                This gives a path-homotopy from $f \cdot \overline{g}$ to $c_p$.

                                ($\Leftarrow$) 
                                To prove that when \( f \sim g \), we can construct a homotopy from \( f \) to \( g \). We define the homotopy \( H(s,t) \) as follows:

                                \[
                                H(s,t) =
                                \begin{cases}
                                K(2s,t) & \text{for } s \in [0,1/2), \\
                                g(2s - 1) & \text{for } s \in [1/2, 1].
                                \end{cases}
                                \]
                                where \( K(s,t) \) is a known homotopy from \( f \cdot \overline{g} \) to \( c_p \).

                                When \( s = 0 \), \( H(0,t) = K(0,t) = p \).
                                When \( s = 1 \), \( H(1,t) = g(2-1) = g(1) = q \).

                                At \( s = 1/2 \), \( H(1/2,t) = K(1,t) = p \), and \( g(2 \times 1/2 - 1) = g(0) = p \), ensuring continuity.

                                When \( t = 0 \), \( H(s,0) \) for \( s \in [0,1/2] \) is \( f\cdot \bar{g}(2s) \), 
                                and for \( s \in [1/2,1] \), it is \( g(2s - 1) \), thus 
                                \( H(s,0) = f \cdot g \cdot \bar{g}\sim f \).
                                When \( t = 1 \), \( H(s,1) = p \), hence \( c_p \cdot g \sim g \).

                                Therefore, \( H(s,t) \) is a homotopy from \( f \) to \( g \), 
                                proving that \( f \sim g \).

                        \item ($\Rightarrow$) 
                                Assume \( X \) is simply connected. Let \( f, g: I \to X \) be two paths from 
                                \( p \) to \( q \). By part (a), \( f \sim g \) if and only if the loop 
                                \( f \cdot \overline{g} \) (the concatenation of \( f \) and the reverse of 
                                \( g \)) is path-homotopic to \( c_p \). Since \( X \) is simply connected, 
                                every loop based at \( p \) is homotopic to \( c_p \). 
                                Thus, \( f \cdot \overline{g} \sim c_p \), and by part (a), \( f \sim g \). 
                                Hence, all paths with the same endpoints are path-homotopic.

                                ($\Leftarrow$) 
                                Conversely, suppose any two paths in \( X \) with the same endpoints are path-homotopic. 
                                Let \( h: I \to X \) be any loop based at \( p \). By assumption, \( h \) is 
                                path-homotopic to the constant path \( c_p \), since both are paths 
                                from \( p \) to \( p \). Thus, \(\pi_1(X, p)\) contains only the trivial element. 
                                Since \( X \) is path-connected and \(\pi_1(X, p)\) is trivial for all \( p \), 
                                \( X \) is simply connected.

                \end{enumerate}
                
                

	\end{solution}
	
	
	
	\begin{problem}
        
        Suppose $f, g : S^n \to S^n$ are continuous maps such that $f(x) \neq -g(x)$ 
        for any $x \in S^n$. Prove that $f$ and $g$ are homotopic.
		
	\end{problem}
	
	\begin{solution}

                We construct an explicit homotopy. For each \( x \in S^n \) and \( t \in [0, 1] \), define the linear interpolation:
                \[
                H(x, t) = \frac{(1 - t)f(x) + t g(x)}{\| (1 - t)f(x) + t g(x) \|}
                \]

                Suppose there exists some \( x \in S^n \) and \( t \in [0, 1] \) such that \( (1 - t)f(x) + t g(x) = 0 \). Then:
                \[
                (1 - t)f(x) = -t g(x)
                \]
                Taking norms gives:
                \[
                (1 - t)\|f(x)\| = t \|g(x)\|
                \]
                Since \( f(x) \) and \( g(x) \) are unit vectors, \( (1 - t) = t \), which implies \( t = \frac{1}{2} \). Substituting back yields \( f(x) = -g(x) \), contradicting the given condition \( f(x) \neq -g(x) \). Thus, the denominator never vanishes.
                
                The numerator and denominator are continuous, and the denominator is non-zero. Hence, \( H \) is continuous.

                Moreover, when \( t = 0 \), \( H(x, 0) = f(x) \). When \( t = 1 \), \( H(x, 1) = g(x) \).

                Therefore, \( H \) defines a homotopy from \( f \) to \( g \), proving that \( f \simeq g \).

	\end{solution}


	\begin{problem}
		Suppose $X$ is a topological space, and $g$ is any path in $X$ from $p$ to $q$. Let 
    \[
    \Phi_g : \pi_1(X, p) \to \pi_1(X, q)
    \]
    denote the group isomorphism defined in Theorem 7.13.

    \begin{itemize}
    \item[(a)] Show that if $h$ is another path in $X$ starting at $q$, then 
    \[
    \Phi_{g \cdot h} = \Phi_h \circ \Phi_g.
    \]

    \item[(b)] Suppose $\psi : X \to Y$ is continuous, and show that the following diagram commutes:
    \[
    \begin{tikzcd}
    \pi_1(X, p) \arrow[r, "\psi_*"] \arrow[d, "\Phi_g"'] & \pi_1(Y, \psi(p)) \arrow[d, "\Phi_{\psi \circ g}"] \\
    \pi_1(X, q) \arrow[r, "\psi_*"'] & \pi_1(Y, \psi(q))
    \end{tikzcd}
    \]
    \end{itemize}


    \end{problem}

    \begin{solution}
        
        \begin{enumerate}[(a)]
                \item Let $[f] \in \pi_1(X, p)$, then 
                \begin{align*}
                \Phi_h \circ \Phi_g[f] &= \Phi_h [\bar{g}]\cdot[f]\cdot[g] = [\bar{h}]\cdot[\bar{g}]\cdot[f]\cdot[g]\cdot[h]\\
                &=[\overline{g\cdot h}]\cdot [f] \cdot [g\cdot h] = \Phi_{g \cdot h}([f])
                \end{align*}
                \item We need to verify the commutative diagram, i.e., \(\psi_* \circ \Phi_g = \Phi_{\psi \circ g} \circ \psi_*\). For any loop \([ \alpha ] \in \pi_1(X, p)\), we calculate the results of both sides of the diagram:

                First, we calculate the left side:
                \[
                \psi_*(\Phi_g([f])) = \psi_*\left( [\bar{g} \cdot f \cdot g] \right) 
                = [\psi \circ (\bar{g}\cdot f \cdot g)].
                \]
                It is equal to
                \[
                [(\psi \circ \bar{g}) \cdot (\psi \circ f) \cdot (\psi \circ g)]
                \]
                
                 Based on the definition of \(\Phi_{\psi \circ g}\), we calculate the right side:
                \[
                \Phi_{\psi \circ g}(\psi_*([f])) = \Phi_{\psi \circ g}([ \psi\circ f ]) = 
                \left[ (\overline{\psi \circ g}) \cdot (\psi\circ f) \cdot (\psi \circ g)\right].
                \]

                The results of both sides are identical, thus the diagram commutes.

        \end{enumerate}

    \end{solution}
\end{document}


