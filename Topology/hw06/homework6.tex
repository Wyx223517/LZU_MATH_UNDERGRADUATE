\documentclass[12pt, a4paper, oneside]{ctexart}
\usepackage{amsmath, amsthm, amssymb, bm, color, framed, graphicx, hyperref, mathrsfs, mathtools, enumerate, tikz}
\usepackage{tikz-cd}
\usetikzlibrary{patterns}


\title{\textbf{Homework 6}}
\author{萃英学院\qquad 2022级\qquad 王一鑫}
\date{\today}
\linespread{1.5}
\newcounter{problemname}
\newenvironment{problem}{\begin{framed}\stepcounter{problemname}\par\noindent\textsc{Problem \arabic{problemname}. }}{\end{framed}\par}
\newenvironment{solution}{%a
	\par\noindent\textsc{Solution. }\ignorespaces
}{%
	\hfill$\qed$\par
}
\newenvironment{note}{\par\noindent\textsc{Note of Problem \arabic{problemname}. }}{\\\par}

\begin{document}
	
	\maketitle
	
	\begin{problem}
        \textbf{Exercise 4.4.} 
        
        Prove that a topological space $X$ is disconnected if and only there exists a nonconstant 
        continuous function from $X$ to the discrete space $\{0, 1\}$.

	
	\end{problem}
	
	\begin{solution}
			
		$\Rightarrow$ If $X$ is disconnected then there exist non-empty open sets $U, V \subset X$ such that $U \cap V = \emptyset$ 
        and $X = U \cup V$. Let $f$ be equal to 0 on $U$ and 1 on $V$. Such $f$ is what we need.

        $\Leftarrow$ Suppose we have such a map $f$ and let $U$ the preimage of 0 and $V$ the preimage of 1. 
        Since $\{0, 1\}$ has the discrete topology the singletons are open hence $U, V$ are open sets in $X$. 
        Then $U \cap V = f^{-1}(\{0\} \cap \{1\}) = \emptyset$. 
        Moreover $X = f^{-1}(\{0, 1\}) = U \cup V$. Therefore $X$ is disconnected.
    
	\end{solution}
	
	\begin{problem}
		4-1. 

        Show that for $n > 1$, $\mathbb{R}^n$ is not homeomorphic to any open subset of $\mathbb{R}$.
        [Hint: if $U \subseteq \mathbb{R}$ is open and $x \in U$, then $U \setminus \{x\}$ is not connected.]

	
	\end{problem}
	
	\begin{solution}

        Suppose for contradiction that there exists a homeomorphism \( f: \mathbb{R}^n \to U \), where \( n > 1 \) and \( U \subseteq \mathbb{R} \) is an open subset. 

        Choose any point \( x \in \mathbb{R}^n \). Since homeomorphisms preserve topological properties
        \( \mathbb{R}^n \setminus \{x\} \) is connected and \( U \setminus \{f(x)\} \) must then also be connected. 

        However for any open \( U \subseteq \mathbb{R} \) and any \( f(x) \in U \), 
        the space \( U \setminus \{f(x)\} \) is not connected. This creates a contradiction.

	\end{solution}
	
	\begin{problem}
		4-4. 
        
        Show that the following topological spaces are not manifolds:

        (a) the union of the x-axis and the y-axis in $\mathbb{R}^2$

        (b) the conical surface $C \subseteq \mathbb{R}^3$ defined by

        \[
        C = \{(x, y, z) : z^2 = x^2 + y^2\}
        \]



	\end{problem}
	
	\begin{solution}
        \begin{enumerate}[(a)]
            \item Suppose $X$ were a $n$-manifold($n\geq 2$). Then there would be a nbhd $U$ of the origin in $X$ that is 
            homeomorphic to $\mathbb{R}^n$. Then we also have that $U$ with the origin removed is homeomorphic 
            to $\mathbb{R}^n$ with one point removed. But this can't be since $U$ without the origin is not 
            connected, whereas $\mathbb{R}^n$ with one point removed is connected.

            Suppose it were a $1$-manifold and $V$ is a nbhd of the origin which is homeomorphic to $\mathbb{R}$. 
            Then removing the origin gives us $4$ components in $V$ and $2$ components in $\mathbb{R}$. So $U$ is also not $1$-manifold.
            \item Similar to (a).

        \end{enumerate}

	
	\end{solution}
	
	
	
	\begin{problem}
		4-5. 
        
        Let $M = \mathbb{S}^1 \times \mathbb{R}$, and let $A = \mathbb{S}^1 \times \{0\}$. 
        Show that the space $M/A$ 
        obtained by collapsing $A$ to a point is homeomorphic to the space $C$ of Problem 4-4(b), 
        and thus is Hausdorff and second countable but not locally Euclidean.


	\end{problem}
	
	\begin{solution}
        Let
        \[
        \begin{tikzcd}
        M \arrow[rd, "f \circ g"] \arrow[d, "g"'] & \\
        M/A \arrow[r, "f"] & C
        \end{tikzcd}
        \]
        Then $f$ is continuous since $g$ and $f\circ g$ are continuous.

        For $\forall(\alpha, \beta, \gamma) \in C$, we have:
        \[
        f\left(\left[ \frac{\alpha}{\sqrt{\alpha^2 + \beta^2}} , \frac{\beta}{\sqrt{\alpha^2 + \beta^2}} ,
        \gamma \right] \right) = (\alpha, \beta, \gamma)
        \]
        Thus $f$ is surjective. Now for $[x], [y] \in M / A$,
        If $f([x]) = f([y]) = (\alpha, \beta, \gamma)$,
        \begin{itemize}
            \item $\gamma = 0$, then $C_x = C_y = 0 \Rightarrow [x] = [y]$.
            \item $\gamma \neq 0$, then $[x] = [(\frac{\alpha}{\gamma}), (\frac{\beta}{\gamma}), \gamma] = [y]\Rightarrow f\text{is injective}$. 
        \end{itemize} 
        Therefore $f$ is homeomorphic.

        Since $M$ is Hausdorff and second countable, then $M / A$ is Hausdorff and second countable.

        According to problem 4-4, $C$ is not a manifold, and $M / A$ is homeomorphic to $C$, 
        then $M / A$ is not locally Euclidean.


	\end{solution}
	
	\begin{problem}
		4-11. 
        
        Let $X$ be a topological space, and let $CX$ be the cone on $X$ (see Example 3.53).

        (a) Show that $CX$ is path-connected.

        (b) Show that $CX$ is locally connected if and only if $X$ is, and locally path-connected if and only if $X$ is.

		

	\end{problem}
	
	\begin{solution}
		
    \textbf{(a)} Let \( CX = (X \times I)\setminus (X \times \{0\})\) and let \( (x_1, y_1), (x_2, y_2) \in CX \), 
    \( f_1, f_2 \in [0,1] \to CX \) such that
    \[
    f_{1}(y) = (x_1, y_1(1 - y)), \quad f_2(y) = (x_2, y_2y).
    \]
    then $f_1$ and $f_2$ are paths. Consider
    \[f_3(y) = \begin{cases}
        f_1(2y), & \text{if } y \in [0, \frac{1}{2}] \\
        f_2(2y), & \text{if } y \in (\frac{1}{2}, 1)
        \end{cases}\]
    Since \( f_1(1) = (x_1, 0) = (x_2, 0) = f_2(1) \) in \( CX \), it is continuous in \( CX \). 
    Moreover $f_3$ connectes \( (x_1, y_1), (x_2, y_2) \in CX \), so \( CX \) is path-connected.

    \textbf{(b)} $\Rightarrow$ Suppose \( CX \) is locally (path-) connected. Then for each \( p \in CX \), 
    there exists a neighborhood \( U_p \) of \( p \), which is (path-) connected. 
    Note that \( X \times \{1\} \cong X \), so \( CX \) is locally (path-) connected.
    Then we can put a projection \( \varphi : CX \to X \), such that \( \varphi(U_p) \) is 
    a (path-) connected neighborhood of \( \varphi(p) \), so \( \alpha \) is (locally path-) connected.


    $\Leftarrow$ Suppose \( X \) is locally (path-) connected. For each \( q \in X \), 
    there exists a neighborhood \( V_q \) of \( q \) which is (path-) connected. $\Rightarrow (V_q \times I)\setminus (V_q \times \{0\})$
    is path-connected. Hence, \( CX \) is locally (path-) connected.

	\end{solution}

\end{document}


