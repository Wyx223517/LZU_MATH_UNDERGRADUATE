\chapter{高阶微分方程}\label{chap:高阶微分方程}
\section{线性微分方程的一般理论}\label{sec:线性微分方程的一般理论}
\begin{definition}[n阶线性微分方程]\label{def:n阶线性微分方程}
    $n$ 阶线性微分方程
\begin{equation}
\frac{d^n x}{dt^n} + a_1(t)\frac{d^{n-1} x}{dt^{n-1}} + \dots + a_{n-1}(t)\frac{dx}{dt} + a_n(t)x = f(t), \label{eq:nth_order_linear_DE_nonhomogeneous}
\end{equation}
其中 $a_i(t)(i=1,2,\dots,n)$ 及 $f(t)$ 都是 $a \le t \le b$ 上的连续函数。
如果 $f(t)=0$,则方程 (\ref{eq:nth_order_linear_DE_nonhomogeneous}) 变为
\begin{equation}
\frac{d^n x}{dt^n} + a_1(t)\frac{d^{n-1} x}{dt^{n-1}} + \dots + a_{n-1}(t)\frac{dx}{dt} + a_n(t)x = 0, \label{eq:nth_order_linear_DE_homogeneous}
\end{equation}
我们称它为 $n$ 阶\textbf{齐次线性微分方程},简称\textbf{齐次线性微分方程},而称一般的方程 (\ref{eq:nth_order_linear_DE_nonhomogeneous}) 为 $n$ 阶\textbf{非齐次线性微分方程},简称\textbf{非齐次线性微分方程},并且通常把方程 (\ref{eq:nth_order_linear_DE_homogeneous}) 叫做对应
于方程 (\ref{eq:nth_order_linear_DE_nonhomogeneous}) 的\textbf{齐次线性微分方程}。
\end{definition}

\begin{theorem}[叠加原理]\label{thm:superposition_principle}
如果 $x_1(t), x_2(t), \dots, x_k(t)$ 是方程 (\ref{eq:nth_order_linear_DE_homogeneous}) 的 $k$ 个解,则它们的
线性组合
$$c_1x_1(t)+c_2x_2(t)+\dots+c_kx_k(t)$$
也是 (\ref{eq:nth_order_linear_DE_homogeneous}) 的解,这里 $c_1,c_2,\dots,c_k$ 是任意常数。
\end{theorem}

\begin{definition}[朗斯基行列式]\label{def:Wronskian}
由定义在 $a \le t \le b$ 上的 $k$ 个可微 $k-1$ 次函数 $x_1(t), x_2(t), \dots, x_k(t)$ 所作成的行
列式
$$W[x_1(t), x_2(t), \dots, x_k(t)] \equiv W(t) = \begin{vmatrix}
x_1(t) & x_2(t) & \cdots & x_k(t) \\
x_1'(t) & x_2'(t) & \cdots & x_k'(t) \\
\vdots & \vdots & & \vdots \\
x_1^{(k-1)}(t) & x_2^{(k-1)}(t) & \cdots & x_k^{(k-1)}(t)
\end{vmatrix}$$
称为这些函数的\textbf{朗斯基行列式}。
\end{definition}

\begin{theorem}[朗斯基行列式与线性相关性]\label{thm:Wronskian_linear_dependence}
若函数 $x_1(t), x_2(t), \dots, x_n(t)$ 在 $a \le t \le b$ 上线性相关,则在 $[a,b]$ 上它们
的朗斯基行列式 $W(t) \equiv 0$.
\end{theorem}
\begin{remark}
    逆定理一般不真。
\end{remark}

\begin{theorem}[朗斯基行列式与线性无关性]\label{thm:Wronskian_linear_independence}
如果方程 (\ref{eq:nth_order_linear_DE_homogeneous}) 的解 $x_1(t), x_2(t), \dots, x_n(t)$ 在 $a \le t \le b$ 上线性无关,则
$W[x_1(t), x_2(t), \dots, x_n(t)]$ 在这个区间的任何点上都不等于零,即 $W(t) \ne 0 (a \le t \le b)$.
\end{theorem}

\begin{theorem}[n阶齐次线性微分方程解的存在性]\label{thm:n_linearly_independent_solutions}
    $n$ 阶齐次线性微分方程 (\ref{eq:nth_order_linear_DE_homogeneous}) 一定存在 $n$ 个线性无关的解。
\end{theorem}


\begin{theorem}[n阶齐次线性微分方程通解结构]\label{thm:general_solution_structure}
    如果 $x_1(t), x_2(t), \dots, x_n(t)$ 是方程 (\ref{eq:nth_order_linear_DE_homogeneous}) 的 $n$ 个线性无关的
解,则方程 (\ref{eq:nth_order_linear_DE_homogeneous}) 的通解可表示为
\begin{equation}
x=c_1x_1(t)+c_2x_2(t)+\dots+c_nx_n(t), \label{eq:general_solution_form}
\end{equation}
其中 $c_1,c_2,\dots,c_n$ 是任意常数,且通解 (\ref{eq:general_solution_form}) 包括了方程 (\ref{eq:nth_order_linear_DE_homogeneous}) 的所有解。
\end{theorem}

\begin{theorem}[n阶非齐次线性微分方程通解结构]\label{thm:general_solution_nonhomogeneous}
设 $x_1(t), x_2(t), \dots, x_n(t)$ 为方程 (\ref{eq:nth_order_linear_DE_homogeneous}) 的基本解组,而 $\bar{x}(t)$ 是方程 (\ref{eq:nth_order_linear_DE_nonhomogeneous}) 的
某一解,则方程 (\ref{eq:nth_order_linear_DE_nonhomogeneous}) 的通解可表示为
\begin{equation}
x=c_1x_1(t)+c_2x_2(t)+\dots+c_nx_n(t)+\bar{x}(t), \label{eq:nonhomogeneous_general_solution}
\end{equation}
其中 $c_1,c_2,\dots,c_n$ 为任意常数,而且这个通解 (\ref{eq:nonhomogeneous_general_solution}) 包括了方程 (\ref{eq:nth_order_linear_DE_nonhomogeneous}) 的所有解。
\end{theorem}
\begin{remark}
    这里可以和线性代数的线性方程组的解联系起来。特解的解法用到常数变易法。
\end{remark}

\begin{example}[常数变易法解非齐次线性微分方程]\label{ex:常数变易法解非齐次线性微分方程}
    设 $x_1(t), x_2(t), \dots, x_n(t)$ 是方程 (\ref{eq:nth_order_linear_DE_homogeneous}) 的基本解组,因而
\begin{equation}
x=c_1x_1(t)+c_2x_2(t)+\dots+c_nx_n(t) \label{eq:homogeneous_solution_repeated}
\end{equation}
为 (\ref{eq:nth_order_linear_DE_homogeneous}) 的通解。把其中的任意常数 $c_i$ 看作 $t$ 的待定函数 $c_i(t) (i=1,2,\dots,n)$,这时
(\ref{eq:homogeneous_solution_repeated}) 变为
\begin{equation}
x=c_1(t)x_1(t)+c_2(t)x_2(t)+\dots+c_n(t)x_n(t). \label{eq:nonhomogeneous_solution_ansatz}
\end{equation}
这里我们想得到一个和$c_i'(t)$有关的方程组,构造如下:
将式 (\ref{eq:nonhomogeneous_solution_ansatz}) 对 $t$ 求导数得
$$x'=c_1'(t)x_1(t)+c_2'(t)x_2(t)+\dots+c_n'(t)x_n(t) + c_1(t)x_1'(t)+c_2(t)x_2'(t)+\dots+c_n(t)x_n'(t).$$
令
\begin{equation}
x_1(t)c_1'(t)+x_2(t)c_2'(t)+\dots+x_n(t)c_n'(t) = 0, \label{eq:condition_1}
\end{equation}
得到
\begin{equation}
x'=c_1(t)x_1'(t)+c_2(t)x_2'(t)+\dots+c_n(t)x_n'(t). \label{eq:x_prime_simplified}
\end{equation}
再将式 (\ref{eq:x_prime_simplified}) 对 $t$ 求导数,并像上面一样做法,令含有函数 $c_i'(t)$ 的部分等于零,我们
又得到一个条件,以此类推,最后我们有含 $n$ 个未知函数 $c_i'(t) (i=1,2,\dots,n)$ 的 $n$ 个方程,
它们组成一个线性代数方程组
\begin{equation}
\left\{
\begin{aligned}
&x_1(t)c_1'(t)+x_2(t)c_2'(t)+\dots+x_n(t)c_n'(t) = 0 \\
&x_1'(t)c_1'(t)+x_2'(t)c_2'(t)+\dots+x_n'(t)c_n'(t) = 0 \\
&\vdots \\
&x_1^{(n-1)}(t)c_1'(t)+x_2^{(n-1)}(t)c_2'(t)+\dots+x_n^{(n-1)}(t)c_n'(t) = f(t)
\end{aligned}
\right. \label{eq:system_of_equations_for_c_prime}
\end{equation}
其系数行列式就是 $W[x_1(t), x_2(t), \dots,
x_n(t)]$, 它不等于零,因而方程组的解唯一确定。

\end{example}

\begin{example}
    求方程 $x''+x=\frac{1}{\cos t}$ 的通解,已知它对应的齐次线性微分方程的基本解组为
$\cos t, \sin t$.
\end{example}
\begin{solution}
    应用常数变易法,令
$$x=c_1(t)\cos t+c_2(t)\sin t.$$
将它代入方程,则可决定 $c_1'(t)$ 和 $c_2'(t)$ 的两个方程
$$ \cos t c_1'(t) + \sin t c_2'(t) = 0 $$
及
$$ -\sin t c_1'(t) + \cos t c_2'(t) = \frac{1}{\cos t} $$
解得
$$c_1'(t) = -\frac{\sin t}{\cos t}, \quad c_2'(t) = 1.$$
由此
$$c_1(t) = \int -\frac{\sin t}{\cos t}\,dt = \ln |\cos t| + \gamma_1,$$
$$c_2(t) = \int 1\,dt = t+\gamma_2.$$
于是原方程的通解为
$$x=\gamma_1\cos t+\gamma_2\sin t+\cos t \ln |\cos t| + t\sin t,$$
其中 $\gamma_1, \gamma_2$ 为任意常数。
\end{solution}

\section{常系数微分方程的解法}\label{sec:常系数微分方程的解法}
对于一般方程的通解是难以求得的,这里考虑常系数微分方程这一特殊情况。
\begin{definition}[常系数齐次线性微分方程]\label{def:常系数齐次线性微分方程}
    设齐次线性微分方程中所有系数都是常数,即方程有如下形状:
\begin{equation}
L[x] \equiv \frac{d^n x}{dt^n} + a_1\frac{d^{n-1} x}{dt^{n-1}} + \dots + a_{n-1}\frac{dx}{dt} + a_nx=0, \label{eq:homogeneous_linear_constant_coeffs}
\end{equation}
其中 $a_1, a_2, \dots, a_n$ 为常数。我们称式 (\ref{eq:homogeneous_linear_constant_coeffs}) 为 $n$ 阶常系数齐次线性微分方程。
\end{definition}

求基本解组的欧拉待定指数函数法,也称作特征值法如下:
\begin{proposition}[特征值法]\label{prop:特征值法}
    注意到
$$L[e^{\lambda t}] = \frac{d^n e^{\lambda t}}{dt^n} + a_1\frac{d^{n-1} e^{\lambda t}}{dt^{n-1}} + \dots + a_{n-1}\frac{d e^{\lambda t}}{dt} + a_n e^{\lambda t}$$
$$= (\lambda^n+a_1\lambda^{n-1}+\dots+a_{n-1}\lambda+a_n)e^{\lambda t} \equiv F(\lambda)e^{\lambda t},$$
其中 $F(\lambda)=\lambda^n+a_1\lambda^{n-1}+\dots+a_{n-1}\lambda+a_n$ 是 $\lambda$ 的 $n$ 次多项式。易知,$x = e^{\lambda t}$ 为方程 (\ref{eq:homogeneous_linear_constant_coeffs}) 的
解的充要条件是 $\lambda$ 是代数方程
\begin{equation}
F(\lambda) = \lambda^n+a_1\lambda^{n-1}+\dots+a_{n-1}\lambda+a_n = 0 \label{eq:characteristic_equation}
\end{equation}
的根。因此,方程 (\ref{eq:characteristic_equation}) 将起着预示方程 (\ref{eq:homogeneous_linear_constant_coeffs}) 的解的特性的作用,我们称它为方程
(\ref{eq:homogeneous_linear_constant_coeffs}) 的\textbf{特征方程};它的根就称为\textbf{特征值}。
\end{proposition}

\begin{proposition}[特征值是单根的情形]\label{prop:特征值是单根的情形}
    设 $\lambda_1, \lambda_2, \dots, \lambda_n$ 是特征方程 (\ref{eq:characteristic_equation}) 的 $n$ 个彼此不相等的根,则对应的方程 (\ref{eq:homogeneous_linear_constant_coeffs})
有如下 $n$ 个解:
\begin{equation}
e^{\lambda_1 t}, e^{\lambda_2 t}, \dots, e^{\lambda_n t}. \label{eq:solutions_distinct_roots}
\end{equation}
这 $n$ 个解在 $a \le t \le b$ 上线性无关,从而组成方程的基本解组。
\end{proposition}

\begin{proof}
    事实上,这时
$$W(t) = \begin{vmatrix}
e^{\lambda_1 t} & e^{\lambda_2 t} & \cdots & e^{\lambda_n t} \\
\lambda_1 e^{\lambda_1 t} & \lambda_2 e^{\lambda_2 t} & \cdots & \lambda_n e^{\lambda_n t} \\
\vdots & \vdots & \ddots & \vdots \\
\lambda_1^{n-1} e^{\lambda_1 t} & \lambda_2^{n-1} e^{\lambda_2 t} & \cdots & \lambda_n^{n-1} e^{\lambda_n t}
\end{vmatrix}$$
$$= e^{(\lambda_1+\lambda_2+\dots+\lambda_n)t} \begin{vmatrix}
1 & 1 & \cdots & 1 \\
\lambda_1 & \lambda_2 & \cdots & \lambda_n \\
\vdots & \vdots & \ddots & \vdots \\
\lambda_1^{n-1} & \lambda_2^{n-1} & \cdots & \lambda_n^{n-1}
\end{vmatrix},$$
从而 $W(t) \ne 0$,于是解组 (\ref{eq:solutions_distinct_roots}) 线性无
关。

\end{proof}
如果 $\lambda_i (i=1,2,\dots,n)$ 均为实数,则 (\ref{eq:solutions_distinct_roots}) 是方程 (\ref{eq:homogeneous_linear_constant_coeffs}) 的 $n$ 个线性无关的实值
解,而方程 (\ref{eq:homogeneous_linear_constant_coeffs}) 的通解可表示为
$$x=c_1e^{\lambda_1 t}+c_2e^{\lambda_2 t}+\dots+c_ne^{\lambda_n t},$$
其中 $c_1,c_2,\dots,c_n$ 为任意常数。

如果特征方程有复根,则因方程的系数是实常数,复根将成对共轭出现。设 $\lambda_1 =
\alpha+i\beta$ 是一特征值,则 $\lambda_2 = \alpha-i\beta$ 也是特征值,因而与这对共轭复根对应的方程 (\ref{eq:homogeneous_linear_constant_coeffs}) 有
两个复值解
$$e^{(\alpha+i\beta)t} = e^{\alpha t}(\cos \beta t+i\sin \beta t),$$
$$e^{(\alpha-i\beta)t} = e^{\alpha t}(\cos \beta t-i\sin \beta t).$$
这样一来,对应该特征方程的一对共轭
复根 $\lambda = \alpha \pm i\beta$, 我们可以求得方程 (\ref{eq:homogeneous_linear_constant_coeffs}) 的两个实值解
$$e^{\alpha t}\cos \beta t, \quad e^{\alpha t}\sin \beta t.$$

\begin{proposition}[特征值有重根的情形]\label{prop:特征值有重根的情形}
假设特征方程 (\ref{eq:characteristic_equation}) 的根 $\lambda_1, \lambda_2, \lambda_3, \dots, \lambda_m$ 的重数依次为 $k_1, k_2, k_3, \dots, k_m$, 且 $k_i \ge 1$
(单根相当于 $k_j=1$),而且 $k_1+k_2+\dots+k_m=n, \lambda_i \ne \lambda_j (i \ne j)$,则方程 (\ref{eq:homogeneous_linear_constant_coeffs}) 对应地有解
\begin{equation}
\begin{cases}
e^{\lambda_1 t}, te^{\lambda_1 t}, \dots, t^{k_1-1}e^{\lambda_1 t}, \\
e^{\lambda_2 t}, te^{\lambda_2 t}, \dots, t^{k_2-1}e^{\lambda_2 t}, \\
\quad \quad \quad \dots \dots \dots \dots \\
e^{\lambda_m t}, te^{\lambda_m t}, \dots, t^{k_m-1}e^{\lambda_m t}.
\end{cases} \label{eq:solutions_other_repeated_roots}
\end{equation}
全体 $n$ 个解构成方程 (\ref{eq:homogeneous_linear_constant_coeffs}) 的基本解组。
\end{proposition}
\begin{remark}
    对于特征方程有复重根的情况,譬如假设 $\lambda=\alpha \pm i\beta$ 是 $k$ 重特征值,则 $\bar{\lambda}=\alpha \mp i\beta$ 也是
$k$ 重特征值,仿命题 \ref{prop:特征值是单根的情形} 一样处理,我们将得到方程 (\ref{eq:homogeneous_linear_constant_coeffs}) 的 $2k$ 个实值解:
$$e^{\alpha t}\cos \beta t, te^{\alpha t}\cos \beta t, t^2e^{\alpha t}\cos \beta t, \dots, t^{k-1}e^{\alpha t}\cos \beta t,$$
$$e^{\alpha t}\sin \beta t, te^{\alpha t}\sin \beta t, t^2e^{\alpha t}\sin \beta t, \dots, t^{k-1}e^{\alpha t}\sin \beta t.$$
\end{remark}

\begin{definition}[常系数非齐次线性微分方程]\label{def:常系数非齐次线性微分方程}
    形如
\begin{equation}
L[x] \equiv \frac{d^n x}{dt^n} + a_1\frac{d^{n-1} x}{dt^{n-1}} + \dots + a_{n-1}\frac{dx}{dt} + a_nx=f(t) \label{eq:nonhomogeneous_linear_constant_coeffs}
\end{equation}
的方程称为常系数非齐次线性微分方程,这里 $a_1,a_2,\dots,a_n$ 是常数,而 $f(t)$ 为连续函数。
\end{definition}

\begin{proposition}[比较系数法的类型1]\label{prop:比较系数法的类型1}
    设 $f(t) = (b_0 t^m + b_1 t^{m-1} + \dots + b_{m-1}t + b_m)e^{\lambda t}$, 其中 $\lambda$ 及 $b_i (i=0,1,\dots,m)$ 为实常数,那么
方程 (\ref{eq:nonhomogeneous_linear_constant_coeffs}) 有形如
\begin{equation}
\tilde{x} = t^k (B_0 t^m + B_1 t^{m-1} + \dots + B_{m-1}t + B_m)e^{\lambda t} \label{eq:particular_solution_form}
\end{equation}
的特解,其中 $k$ 为特征方程 $F(\lambda)=0$ 的根 $\lambda$ 的重数 (单根相当于 $k=1$;当 $\lambda$ 不是特征值
时,取 $k=0$),而 $B_0, B_1, \dots, B_m$ 是待定常数,可以通过将 (\ref{eq:particular_solution_form}) 代入微分方程并比较 $t$ 的
同次幂系数的方法确定。
\end{proposition}

\begin{example}
    求 $\frac{d^3 x}{dt^3}+3\frac{d^2 x}{dt^2}+3\frac{dx}{dt}+x=e^{-t}(t-5)$ 的通解。
\end{example}
\begin{solution}
    特征方程 $\lambda^3+3\lambda^2+3\lambda+1=(\lambda+1)^3=0$ 有三重根 $\lambda_{1,2,3}=-1$, 对应齐次方程的通
解为
$$x=(c_1+c_2t+c_3t^2)e^{-t},$$
其中 $c_1,c_2,c_3$ 为任意常数,且方程有形式为 $\tilde{x} = t^3 (A+Bt) e^{-t}$ 的特解。将它代入方程得
$$(6A+24Bt)e^{-t} = e^{-t}(t-5),$$
比较系数求得 $A=-\frac{5}{6}, B=\frac{1}{24}$。从而 $\tilde{x} = \frac{1}{24}t^3(t-20)e^{-t}$. 故方程的通解为
$$x=(c_1+c_2t+c_3t^2)e^{-t} + \frac{1}{24}t^3(t-20)e^{-t}.$$
\end{solution}

\begin{proposition}[比较系数法的类型2]\label{prop:比较系数法的类型2}
    设 $f(t) = [A(t)\cos \beta t+B(t)\sin \beta t]e^{\alpha t}$, 其中 $\alpha, \beta$ 为常数,而 $A(t), B(t)$ 是带实系数
的 $t$ 的多项式,其中一个的次数为 $m$,而另一个的次数不超过 $m$,那么我们有如下结论:
方程 (\ref{eq:nonhomogeneous_linear_constant_coeffs}) 有形如
\begin{equation}
\tilde{x} = t^k [P(t)\cos \beta t+Q(t)\sin \beta t]e^{\alpha t} \label{eq:particular_solution_trig_form}
\end{equation}
的特解,这里 $k$ 为特征方程 $F(\lambda)=0$ 的根 $\alpha+i\beta$ 的重数,而 $P(t),Q(t)$ 均为待定的带实
系数的次数不高于 $m$ 的多项式,可以通过比较系数的方法来确定。

注意,正确写出特解形式是待定系数法的关键,在此类型的求解过程中应把 $P(t)$,
$Q(t)$ 均假设为 $m$ 次完全多项式来实际演算。
\end{proposition}

\begin{example}
    求方程 $\frac{d^2 x}{dt^2}+4\frac{dx}{dt}+4x=\cos 2t$ 的通解。
\end{example}
\begin{solution}
    特征方程 $\lambda^2+4\lambda+4=0$ 有重根 $\lambda_1=\lambda_2=-2$, 因此对应齐次线性微分方程的
通解为
$$x=(c_1+c_2t)e^{-2t},$$
其中 $c_1,c_2$ 为任意常数。现求非齐次线性微分方程的一个特解。因为 $\pm 2i$ 不是特征值,我
们求形如 $\tilde{x}=A\cos 2t+B\sin 2t$ 的特解,将它代入原方程并简化得到
$$8B\cos 2t-8A\sin 2t = \cos 2t,$$
比较同类项系数得 $A=0, B=\frac{1}{8}$,从而 $\tilde{x} = \frac{1}{8}\sin 2t$. 因此,原方程的通解为
$$x=(c_1+c_2t)e^{-2t} + \frac{1}{8}\sin 2t.$$
\end{solution}

\section{高阶微分方程的降阶与幂级数解法}\label{sec:高阶微分方程的降阶与幂级数解法}
$n$ 阶微分方程一般可写为
$$F(t,x,x',\dots,x^{(n)})=0.$$
下面讨论三类特殊方程的降阶问题。

\begin{example}[第一类可降阶的方程]\label{ex:第一类可降阶的方程}
    方程不显含未知函数 $x$, 或更一般地,设方程不含 $x,x',\dots,x^{(k-1)}$,即方程形如
\begin{equation}
F(t,x^{(k)},x^{(k+1)},\dots,x^{(n)})=0 \quad (1 \le k \le n). \label{eq:ODE_missing_x_to_xk_minus_1}
\end{equation}
若令 $x^{(k)}=y$,则方程即降为关于 $y$ 的 $n-k$ 阶方程
\begin{equation}
F(t,y,y',\dots,y^{(n-k)})=0. \label{eq:ODE_reduced_order_y}
\end{equation}
如果能够求得方程 (\ref{eq:ODE_reduced_order_y}) 的通解
$$y=\varphi(t,c_1,c_2,\dots,c_{n-k}),$$
即
$$x^{(k)}=\varphi(t,c_1,c_2,\dots,c_{n-k}),$$
再经过 $k$ 次积分得到
$$x=\psi(t,c_1,c_2,\dots,c_n),$$
其中 $c_1,c_2,\dots,c_n$ 为任意常数。可以验证,这就是方程 (\ref{eq:ODE_missing_x_to_xk_minus_1}) 的通解。
特别地,若二阶方程不显含 $x$ (相当于 $n=2,k=1$ 的情形),则用变换 $x'=y$ 便把方
程化为一阶方程。
\end{example}

\begin{example}[第二类可降阶的方程]\label{ex:第二类可降阶的方程}
    不显含自变量 $t$ 的方程
\begin{equation}
F(x,x',\dots,x^{(n)})=0. \label{eq:ODE_missing_t}
\end{equation}
我们指出,若令 $x'=y$, 并以它为新未知函数,而视 $x$ 为新自变量,则方程就可降低一阶。

事实上,在所作的假定下,$x'=y, x''=\frac{dy}{dt}=\frac{dy}{dx}\frac{dx}{dt}=y\frac{dy}{dx}, x'''=\frac{d}{dx}\left(y\frac{dy}{dx}\right)\frac{dy}{dx}+y^2\frac{d^2y}{dx^2}, \dots$,采用数
学归纳法不难证明,$x^{(k)}$ 可用 $y, \frac{dy}{dx}, \dots, \frac{d^{k-1}y}{dx^{k-1}}$ 表示 ($k \le n$)。将这些表达式代入式 (\ref{eq:ODE_missing_t}) 就
得到
$$G\left(x,y,\frac{dy}{dx},\dots,\frac{d^{n-1}y}{dx^{n-1}}\right)=0,$$
这是关于 $x,y$ 的 $n-1$ 阶方程,比原方程 (\ref{eq:ODE_missing_t}) 低一阶。
\end{example}

\begin{example}[二阶线性微分方程的幂级数解法]\label{ex:二阶线性微分方程的幂级数解法}
    求方程 $y''-2xy'-4y=0$ 的满足初值条件 $y(0)=0$ 及 $y'(0)=1$ 的解。
\end{example}

\begin{solution}
    设式 \begin{equation}
y=a_0+a_1x+a_2x^2+\dots+a_nx^n+\dots \label{eq:power_series_solution_ansatz}
\end{equation}为方程的解,利用初值条件,可以得到
$$a_0=0, \quad a_1=1,$$
因而
\begin{align*}
y&=x+a_2x^2+a_3x^3+\dots+a_nx^n+\dots, \\
y'&=1+2a_2x+3a_3x^2+\dots+na_nx^{n-1}+\dots, \\
y''&=2a_2+3 \cdot 2a_3x+\dots+n(n-1)a_nx^{n-2}+\dots.
\end{align*}
将 $y,y',y''$ 的表达式代入原方程,合并 $x$ 的同次幂的项,并令各项系数等于零,得到
$$a_2=0, a_3=1, a_4=0,\dots, a_n=\frac{2}{n-1}a_{n-2}, \dots$$
因而
$$a_5=\frac{1}{2!}, a_6=0, a_7=\frac{1}{3!}, a_8=0, a_9=\frac{1}{4!}, \dots$$
最后得
$$a_{2k+1} = \frac{1}{k!}, \quad a_{2k}=0.$$
对一切正整数 $k$ 成立。
将 $a_i (i=0,1,2,\dots)$ 的值代回式 (\ref{eq:power_series_solution_ansatz}) 就得到
$$y=x+x^3+\frac{x^5}{2!}+\dots+\frac{x^{2k+1}}{k!}+\dots = x\left(1+\frac{x^2}{1!}+\frac{x^4}{2!}+\dots+\frac{x^{2k}}{k!}+\dots\right) = xe^{x^2}.$$
这就是方程的满足所给初值条件的解。
\end{solution}
