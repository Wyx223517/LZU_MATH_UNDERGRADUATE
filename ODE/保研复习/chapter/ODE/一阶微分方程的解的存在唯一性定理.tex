\chapter{一阶微分方程的解的存在定理}\label{chap:一阶微分方程的解的存在唯一性定理}
复旦面试的惨痛教训。。。
\section{解的存在唯一性定理}
\begin{definition}[Lipschitz条件]\label{def:Lipschitz条件}
    函数 $f(x,y)$ 称为在 $R$ 上关于 $y$ 满足\textbf{利普希茨(Lipschitz)条件},如果存在常数 $L>0$,
使得不等式
$$|f(x,y_1)-f(x,y_2)| \le L|y_1-y_2|$$
对于所有 $(x,y_1), (x,y_2) \in R$ 都成立,$L$ 称为\textbf{利普希茨常数}。
\end{definition}
首先考虑导数已解出的一阶微分方程
\begin{equation}
\frac{dy}{dx} = f(x,y), \label{eq:first_order_DE_dy_dx_fxy}
\end{equation}
这里 $f(x,y)$ 是在矩形域
\begin{equation}
R: |x-x_0| \le a, |y-y_0| \le b \label{eq:rectangular_domain}
\end{equation}
上的连续函数。
\begin{theorem}[Picard定理]\label{thm:existence_uniqueness}
    如果 $f(x,y)$ 在矩形域 $R$ 上连续且关于 $y$ 满足利普希茨条件,则方程 (\ref{eq:first_order_DE_dy_dx_fxy})
存在唯一的解 $y=\varphi(x)$, 定义于 $|x-x_0| \le h$ 上,连续且满足初值条件
\begin{equation}
\varphi(x_0)=y_0, \label{eq:initial_condition}
\end{equation}
这里 $h=\min\left(a, \frac{b}{M}\right)$, $M= \max_{(x,y) \in R} |f(x,y)|$.
\end{theorem}

下面的讨论仅针对于 $x_0\le x \le x_0+h$ 进行,另一侧区域同理。

\begin{proposition}[积分方程与微分方程等价]\label{prop:integral_equation_equivalence}
    设 $y=\varphi(x)$ 是方程 (\ref{eq:first_order_DE_dy_dx_fxy}) 的定义于 $x_0 \le x \le x_0+h$ 上,满足初值条件
$$\varphi(x_0)=y_0$$
的解,则 $y=\varphi(x)$ 是积分方程
\begin{equation}
y=y_0 + \int_{x_0}^{x} f(x,\varphi(x))\,dx, \quad x_0 \le x \le x_0+h \label{eq:integral_equation}
\end{equation}
的定义于 $x_0 \le x \le x_0+h$ 上的连续解。反之亦然。
\end{proposition}
\begin{proof}
    因为 $y=\varphi(x)$ 是方程 (\ref{eq:first_order_DE_dy_dx_fxy}) 的解, 故有
$$\frac{d\varphi(x)}{dx}=f(x,\varphi(x)),$$
两边从 $x_0$ 到 $x$ 取定积分,得到
$$\varphi(x)-\varphi(x_0) = \int_{x_0}^{x} f(x,\varphi(x))\,dx, \quad x_0 \le x \le x_0+h.$$
把 $\varphi(x_0)=y_0$ 代入上式,即有
$$\varphi(x)=y_0 + \int_{x_0}^{x} f(x,\varphi(x))\,dx, \quad x_0 \le x \le x_0+h,$$
因此,$y=\varphi(x)$ 是式 (\ref{eq:integral_equation}) 的定义于 $x_0 \le x \le x_0+h$ 上的连续解。

反之,如果 $y=\varphi(x)$ 是式 (\ref{eq:integral_equation}) 的连续解,则有
\begin{equation}
\varphi(x)=y_0 + \int_{x_0}^{x} f(x,\varphi(x))\,dx, \quad x_0 \le x \le x_0+h. \label{eq:integral_equation_repeated}
\end{equation}
对 $x$ 求导数,得到
$$\frac{d\varphi(x)}{dx}=f(x,\varphi(x)),$$
又把 $x=x_0$ 代入式 (\ref{eq:integral_equation_repeated}),得到
$$\varphi(x_0)=y_0.$$
因此,$y=\varphi(x)$ 是方程 (\ref{eq:first_order_DE_dy_dx_fxy}) 的定义于 $x_0 \le x \le x_0+h$ 上,且满足初值条件 $\varphi(x_0)=y_0$ 的解。
\end{proof}

现在取 $\varphi_0(x)=y_0$, 构造皮卡逐步逼近函数序列如下:
\begin{equation}
\begin{cases} \varphi_0(x) = y_0, \\ \varphi_n(x) = y_0 + \int_{x_0}^{x} f(\xi, \varphi_{n-1}(\xi))\,d\xi, \quad x_0 \le x \le x_0+h \quad (n=1,2,\dots). \end{cases} \label{eq:Picard_iteration}
\end{equation}

\begin{proposition}[Picard逼近函数序列的良定义性]\label{prop:Picard_boundedness}
    对于所有的 $n$, 式 (\ref{eq:Picard_iteration}) 中函数 $\varphi_n(x)$ 在 $x_0 \le x \le x_0+h$ 上有定义、连续且满足
不等式
\begin{equation}
|\varphi_n(x)-y_0| \le b. \label{eq:Picard_bound_b}
\end{equation}
\end{proposition}
\begin{proof}
    当 $n=1$ 时, $\varphi_1(x)=y_0 + \int_{x_0}^{x} f(\xi,y_0)\,d\xi$. 显然 $\varphi_1(x)$ 在 $x_0 \le x \le x_0+h$ 上有定义、
连续且有
$$|\varphi_1(x) - y_0| = \left|\int_{x_0}^{x} f(\xi,y_0)\,d\xi\right| \le \int_{x_0}^{x} |f(\xi,y_0)|\,d\xi \le M(x-x_0) \le Mh \le b,$$
即命题 \ref{prop:Picard_boundedness} 当 $n=1$ 时成立。现在我们用数学归纳法证明对于任何正整数 $n$, 命题 \ref{prop:Picard_boundedness} 都成
立。为此,设命题 \ref{prop:Picard_boundedness} 当 $n=k$ 时成立,也即 $\varphi_k(x)$ 在 $x_0 \le x \le x_0+h$ 上有定义、连续且满足不
等式 $|\varphi_k(x)-y_0| \le b$, 这时
$$\varphi_{k+1}(x)=y_0 + \int_{x_0}^{x} f(\xi,\varphi_k(\xi))\,d\xi.$$
由假设,命题 \ref{prop:Picard_boundedness} 当 $n=k$ 时 $\varphi_k(x)$ 在 $x_0 \le x \le x_0+h$ 上有定义、连续且有
$$|\varphi_{k+1}(x)-y_0| \le \left|\int_{x_0}^{x} f(\xi,\varphi_k(\xi))\,d\xi\right| \le M(x-x_0) \le Mh \le b,$$
即命题 \ref{prop:Picard_boundedness} 当 $n=k+1$ 时也成立。由数学归纳法知命题 \ref{prop:Picard_boundedness} 对于所有 $n$ 均成立。命题 \ref{prop:Picard_boundedness} 证毕。
\end{proof}

\begin{proposition}[Picard逼近函数序列的一致收敛性]\label{prop:Picard_uniform_convergence}
    函数序列 $\{\varphi_n(x)\}$ 在 $x_0 \le x \le x_0+h$ 上是一致收敛的。
\end{proposition}
\begin{proof}
    我们考虑级数
\begin{equation}
\varphi_0(x) + \sum_{k=1}^\infty [\varphi_k(x)-\varphi_{k-1}(x)], \quad x_0 \le x \le x_0+h \label{eq:Picard_series}
\end{equation}
它的部分和为
$$\varphi_0(x) + \sum_{k=1}^n [\varphi_k(x)-\varphi_{k-1}(x)] = \varphi_n(x).$$
因此,要证明函数序列 $\{\varphi_n(x)\}$ 在 $x_0 \le x \le x_0+h$ 上一致收敛,只需证明级数 (\ref{eq:Picard_series}) 在 $x_0 \le x \le x_0+h$ 上一致收敛。为此,我们进行如下的估计,由 \eqref{eq:Picard_iteration} 有
\begin{equation}
|\varphi_1(x)-\varphi_0(x)| = \left|\int_{x_0}^{x} f(\xi,\varphi_0(\xi))\,d\xi\right| \le M(x-x_0). \label{eq:phi1_minus_phi0_bound}
\end{equation}
及
$$|\varphi_2(x)-\varphi_1(x)| = \left|\int_{x_0}^{x} [f(\xi,\varphi_1(\xi))-f(\xi,\varphi_0(\xi))]\,d\xi\right|.$$
利用利普希茨条件及式 (\ref{eq:phi1_minus_phi0_bound}),得到
$$|\varphi_2(x)-\varphi_1(x)| \le L \int_{x_0}^{x} |\varphi_1(\xi)-\varphi_0(\xi)|\,d\xi \le L \int_{x_0}^{x} M(\xi-x_0)\,d\xi = \frac{ML}{2!}(x-x_0)^2.$$
设对于正整数 $n$, 不等式
$$|\varphi_n(x)-\varphi_{n-1}(x)| \le \frac{ML^{n-1}}{n!}(x-x_0)^n$$
成立,则由利普希茨条件,当 $x_0 \le x \le x_0+h$ 时,有
\begin{align*}
|\varphi_{n+1}(x)-\varphi_n(x)| &= \left|\int_{x_0}^{x} [f(\xi,\varphi_n(\xi))-f(\xi,\varphi_{n-1}(\xi))]\,d\xi\right| \\
&\le L \int_{x_0}^{x} |\varphi_n(\xi)-\varphi_{n-1}(\xi)|\,d\xi \\
&\le L \int_{x_0}^{x} \frac{ML^n}{n!}(\xi-x_0)^n\,d\xi \\ % Typo in original image: ML^(n-1) -> ML^n
&= \frac{ML^n}{(n+1)!}(x-x_0)^{n+1}.
\end{align*}
于是,由数学归纳法,对于所有的正整数 $k$, 有如下的估计:
\begin{equation}
|\varphi_k(x)-\varphi_{k-1}(x)| \le \frac{ML^{k-1}}{k!}(x-x_0)^k, \quad x_0 \le x \le x_0+h. \label{eq:phi_k_diff_bound_x}
\end{equation}
从而可知,当 $x_0 \le x \le x_0+h$ 时,
\begin{equation}
|\varphi_k(x)-\varphi_{k-1}(x)| \le \frac{ML^{k-1}}{k!}h^k. \label{eq:phi_k_diff_bound_h}
\end{equation}
式 (\ref{eq:phi_k_diff_bound_h}) 的右端是正项级数
$$\sum_{k=1}^\infty \frac{ML^{k-1}}{k!}h^k$$
的一般项。由魏尔斯特拉斯 (Weierstrass) 判别法,级数 (\ref{eq:Picard_series}) 在 $x_0 \le x \le x_0+h$ 上一致收敛,因而序列 $\{\varphi_n(x)\}$ 也在 $x_0 \le x \le x_0+h$ 上一致收敛。命题 \ref{prop:Picard_uniform_convergence} 证毕。

现设
$$\lim_{n\to\infty} \varphi_n(x) = \varphi(x),$$
则 $\varphi(x)$ 也在 $x_0 \le x \le x_0+h$ 上连续,且由 (\ref{eq:Picard_bound_b}) 可知
$$|\varphi(x)-y_0| \le b.$$
\end{proof}

\begin{proposition}[函数序列收敛项是原方程的解]\label{prop:Picard_limit_is_solution}
    $\varphi(x)$ 是积分方程 (\ref{eq:integral_equation}) 的定义于 $x_0 \le x \le x_0+h$ 上的连续解。
\end{proposition}
\begin{proof}
    由利普希茨条件
$$|f(x,\varphi_n(x))-f(x,\varphi(x))| \le L|\varphi_n(x)-\varphi(x)|$$
以及 $\{\varphi_n(x)\}$ 在 $x_0 \le x \le x_0+h$ 上一致收敛于 $\varphi(x)$, 即序列 $\{f(x,\varphi_n(x))\}$ 在 $x_0 \le x \le
x_0+h$ 上一致收敛于 $f(x,\varphi(x))$。因而,对式 (\ref{eq:Picard_iteration}) 两边取极限,得到
$$\lim_{n\to\infty} \varphi_n(x)=y_0+\lim_{n\to\infty}\int_{x_0}^{x} f(\xi,\varphi_{n-1}(\xi))\,d\xi = y_0+\int_{x_0}^{x} \lim_{n\to\infty}f(\xi,\varphi_{n-1}(\xi))\,d\xi,$$
即
$$\varphi(x)=y_0 + \int_{x_0}^{x} f(\xi,\varphi(\xi))\,d\xi.$$
这就是说,$\varphi(x)$ 是积分方程 (\ref{eq:integral_equation}) 的定义于 $x_0 \le x \le x_0+h$ 上的连续解。
\end{proof}

\begin{proposition}[唯一性]\label{prop:Picard_uniqueness}

设 $\psi(x)$ 是积分方程 (\ref{eq:integral_equation}) 的定义于 $x_0 \le x \le x_0+h$ 上的另一个连续解,则
$\varphi(x) = \psi(x)$ ($x_0 \le x \le x_0+h$)。

\end{proposition}

\begin{proof}
    我们证明 $\psi(x)$ 也是序列 $\{\varphi_n(x)\}$ 的一致收敛极限函数。为此,从
$$\varphi_0(x) = y_0,$$
$$\varphi_n(x) = y_0 + \int_{x_0}^{x} f(\xi,\varphi_{n-1}(\xi))\,d\xi \quad (n \ge 1),$$
以及
$$\psi(x) = y_0 + \int_{x_0}^{x} f(\xi,\psi(\xi))\,d\xi$$
可以进行如下估计:
$$|\varphi_0(x)-\psi(x)| = \left|\int_{x_0}^{x} f(\xi,\psi(\xi))\,d\xi\right| \le M(x-x_0),$$
\begin{align*}
|\varphi_1(x)-\psi(x)| &= \left|\int_{x_0}^{x} [f(\xi,\varphi_0(\xi))-f(\xi,\psi(\xi))]\,d\xi\right| \\
&\le L \int_{x_0}^{x} |\varphi_0(\xi)-\psi(\xi)|\,d\xi \\
&\le L \int_{x_0}^{x} M(\xi-x_0)\,d\xi = \frac{ML}{2!}(x-x_0)^2.
\end{align*}
现设 $|\varphi_n(x)-\psi(x)| \le \frac{ML^n}{n!}(x-x_0)^n$, 则有
\begin{align*}
|\varphi_{n+1}(x)-\psi(x)| &= \left|\int_{x_0}^{x} [f(\xi,\varphi_n(\xi))-f(\xi,\psi(\xi))]\,d\xi\right| \\
&\le L \int_{x_0}^{x} |\varphi_n(\xi)-\psi(\xi)|\,d\xi \\
&\le L \int_{x_0}^{x} \frac{ML^n}{n!}(\xi-x_0)^n\,d\xi \\
&= \frac{ML^n}{(n+1)!}(x-x_0)^{n+1}.
\end{align*}
故由数学归纳法知,对于所有的正整数 $n$, 有估计式
\begin{equation}
|\varphi_n(x)-\psi(x)| \le \frac{ML^n}{(n+1)!}(x-x_0)^{n+1}. \label{eq:phi_n_psi_diff_x}
\end{equation}
因此,在 $x_0 \le x \le x_0+h$ 上有
\begin{equation}
|\varphi_n(x)-\psi(x)| \le \frac{ML^n}{(n+1)!}h^{n+1}. \label{eq:phi_n_psi_diff_h}
\end{equation}
$\frac{ML^n}{(n+1)!}h^{n+1}$ 是收敛级数的一般项,故当 $n \to \infty$ 时,$\frac{ML^n}{(n+1)!}h^{n+1} \to 0$. 因而 $\{\varphi_n(x)\}$ 在
$x_0 \le x \le x_0+h$ 上一致收敛于 $\psi(x)$。根据极限的唯一性,即得
$$\varphi(x) = \psi(x), \quad x_0 \le x \le x_0+h.$$
命题 \ref{prop:Picard_uniqueness} 证毕。
\end{proof}

\begin{remark}
    命题 \ref{prop:integral_equation_equivalence} - 命题 \ref{prop:Picard_uniqueness} 证明了定理 \ref{thm:existence_uniqueness}。主要思路就是先证明微分方程与积分方程的等价性,再基于此构造Picard迭代序列,证明它的良定义性与唯一性,最后可以得到原方程的解,再证明唯一性。
\end{remark}

\section{解的延拓}\label{sec:解的延拓}
\begin{theorem}[解的延拓定理]\label{thm:solution_extension_theorem}
如果方程 (\ref{eq:first_order_DE_dy_dx_fxy}) 右端的函数 $f(x,y)$ 在有界区域 $G$ 中连
续,且在 $G$ 内关于 $y$ 满足局部的利普希茨条件,那么方程 (\ref{eq:first_order_DE_dy_dx_fxy}) 的通过 $G$ 内任何一点
$(x_0,y_0)$ 的解 $y=\varphi(x)$ 可以延拓,直到点 $(x,\varphi(x))$ 任意接近区域 $G$ 的边界。以向 $x$ 增大
的一方的延拓来说,如果 $y=\varphi(x)$ 只能延拓到区间 $[x_0,d)$ 上,则当 $x \to d$ 时,$(x,\varphi(x))$
趋于区域 $G$ 的边界。
\end{theorem}