\chapter{一阶微分方程的初等解法}\label{chap:一阶微分方程的初等解法}
\section{变量分离}\label{sec:变量分离}
\begin{definition}[变量分离方程]\label{def:变量分离方程}
    形如
\begin{equation}
\frac{dy}{dx} = f(x)\varphi(y) \label{eq:variable_separation_form}
\end{equation}
的方程,称为\textbf{变量分离方程},这里 $f(x), \varphi(y)$ 分别是 $x,y$ 的连续函数。
\end{definition}

\begin{solution}
    如果 $\varphi(y) \ne 0$, 我们可以将式 (\ref{eq:variable_separation_form}) 改写成
$$\frac{dy}{\varphi(y)} = f(x) dx,$$
这样变量就“分离”开了,然后两边积分,得
\begin{equation}
\int \frac{dy}{\varphi(y)} = \int f(x) dx + c. \label{eq:separated_variables_integrated}
\end{equation}
这里我们把积分常数 $c$ 明确写出来,把 $\int \frac{dy}{\varphi(y)}, \int f(x) dx$ 分别理解为 $\frac{1}{\varphi(y)}, f(x)$ 的原函
数。常数 $c$ 的取值必须保证式 (\ref{eq:separated_variables_integrated}) 有意义。
\end{solution}

\begin{definition}[齐次微分方程]\label{def:齐次微分方程}
    形如
\begin{equation}
\frac{dy}{dx} = g\left(\frac{y}{x}\right) \label{eq:homogeneous_DE}
\end{equation}
的方程称为\textbf{齐次微分方程},这里 $g(u)$ 是 $u$ 的连续函数。

\end{definition}
\begin{solution}
    作变量变换
\begin{equation}
u = \frac{y}{x}, \label{eq:homogeneous_substitution_u}
\end{equation}
即 $y=ux$,两边对 $x$ 求导数,得
\begin{equation}
\frac{dy}{dx} = x\frac{du}{dx} + u. \label{eq:homogeneous_dy_dx}
\end{equation}
将式 (\ref{eq:homogeneous_substitution_u}),式 (\ref{eq:homogeneous_dy_dx}) 代入式 (\ref{eq:homogeneous_DE}),并整理得
\begin{equation}
\frac{du}{dx} = \frac{g(u)-u}{x}. \label{eq:homogeneous_transformed_DE}
\end{equation}
方程 (\ref{eq:homogeneous_transformed_DE}) 是一个变量分离的微分方程,可按照式 \eqref{eq:separated_variables_integrated} 的方法求解,然后将 $u=\frac{y}{x}$ 代
回,便得式 (\ref{eq:homogeneous_DE}) 的解。
\end{solution}

\begin{definition}[可化为变量分离方程的类型二]\label{def:case2}
    形如
\begin{equation}
\frac{dy}{dx} = f(ax+by+c) \quad (a \ne 0, b \ne 0), \label{eq:linear_argument_DE}
\end{equation}
这里 $f(u)$ 是 $u$ 的连续函数。

\end{definition}
\begin{solution}
    作变量变换
\begin{equation}
u=ax+by+c, \label{eq:linear_argument_substitution_u}
\end{equation}
两边对 $x$ 求导数,得
\begin{equation}
\frac{du}{dx} = a+b\frac{dy}{dx}. \label{eq:linear_argument_du_dx}
\end{equation}
代入方程 (\ref{eq:linear_argument_DE}),得
\begin{equation}
\frac{du}{dx} = a+bf(u). \label{eq:linear_argument_transformed_DE}
\end{equation}
方程 (\ref{eq:linear_argument_transformed_DE}) 是一个变量分离的微分方程,可按照式 \eqref{eq:separated_variables_integrated} 的方法求解,然后将 $u=ax+by+c$
代回,便得式 (\ref{eq:linear_argument_DE}) 的解。
\end{solution}

\begin{definition}[可化为变量分离方程的类型二]\label{def:case3}
  形如  \begin{equation}
\frac{dy}{dx} = f\left(\frac{a_1x+b_1y+c_1}{a_2x+b_2y+c_2}\right) \label{eq:general_linear_fraction_DE}
\end{equation}
的方程也可以经变量变换化为变量分离方程,这里 $a_1,b_1,c_1,a_2,b_2,c_2$ 均为常数。当这
些常数取某些特殊值时,式 (\ref{eq:general_linear_fraction_DE}) 可为式 (\ref{eq:homogeneous_DE}) 或式 (\ref{eq:linear_argument_DE}) 的情形。
\end{definition}
\begin{solution}
    当 $c_1=c_2=0$ 时,$\frac{dy}{dx} = f\left(\frac{a_1x+b_1y}{a_2x+b_2y}\right) = g\left(\frac{y}{x}\right)$,即为齐次微分方程 (\ref{eq:homogeneous_DE})。因此这里主要
讨论 $c_1,c_2$ 不全为零时的情形。这时方程 (\ref{eq:general_linear_fraction_DE}) 右端分子分母都是 $x,y$ 的一次多项式,
因此
\begin{equation}
\begin{cases} a_1x+b_1y+c_1=0, \\ a_2x+b_2y+c_2=0 \end{cases} \label{eq:system_linear_equations}
\end{equation}
表示 $Oxy$ 平面上的两条相交直线。针对方程组的系数行列式 $\begin{vmatrix} a_1 & b_1 \\ a_2 & b_2 \end{vmatrix}$,分两种情况进
行讨论。

\begin{enumerate}
    \item $\begin{vmatrix} a_1 & b_1 \\ a_2 & b_2 \end{vmatrix} \ne 0$ 的情形,此时 $\frac{a_1}{a_2} \ne \frac{b_1}{b_2}$
这时方程组 (\ref{eq:system_linear_equations}) 有唯一解(即 $Oxy$ 平面上两条直线的交点为 $(\alpha, \beta)$)。若令
\begin{equation}
\begin{cases} X = x-\alpha, \\ Y = y-\beta, \end{cases} \label{eq:translation_variables}
\end{equation}
则式 (\ref{eq:system_linear_equations}) 化为
\begin{equation}
\begin{cases} a_1X+b_1Y=0, \\ a_2X+b_2Y=0. \end{cases} \label{eq:translated_system}
\end{equation}
从而式 (\ref{eq:general_linear_fraction_DE}) 化为
\begin{equation}
\frac{dY}{dX} = f\left(\frac{a_1X+b_1Y}{a_2X+b_2Y}\right) = g\left(\frac{Y}{X}\right), \label{eq:transformed_homogeneous_DE_new_vars}
\end{equation}
这是一个齐次微分方程,求解代回原变量即可得原方程 (\ref{eq:general_linear_fraction_DE}) 的解。
    \item $\begin{vmatrix} a_1 & b_1 \\ a_2 & b_2 \end{vmatrix} = 0$ 的情形
要使此行列式等于 0,不外乎三种情形。
\begin{enumerate}
    \item[(i)] $a_1=0, b_1=0$ 时,式 (\ref{eq:general_linear_fraction_DE}) 变为 $\frac{dy}{dx}=f\left(\frac{c_1}{a_2x+b_2y+c_2}\right)$;而 $a_2=0, b_2=0$ 时,式 (\ref{eq:general_linear_fraction_DE}) 变为 $\frac{dy}{dx}=f\left(\frac{a_1x+b_1y+c_1}{c_2}\right)$. 这时式 (\ref{eq:general_linear_fraction_DE}) 为式 (\ref{eq:linear_argument_DE}) 的情形。
    \item[(ii)] $b_1=0, b_2=0$ 时,式 (\ref{eq:general_linear_fraction_DE}) 变为 $\frac{dy}{dx}=f\left(\frac{a_1x+c_1}{a_2x+c_2}\right)$;而 $a_1=0, a_2=0$ 时,式 (\ref{eq:general_linear_fraction_DE}) 变为 $\frac{dy}{dx}=f\left(\frac{b_1y+c_1}{b_2y+c_2}\right)$. 这时式 (\ref{eq:general_linear_fraction_DE}) 可化为变量分离方程 (\ref{eq:variable_separation_form})。
    \item[(iii)] $\frac{a_1}{a_2} = \frac{b_1}{b_2} = k$ 时,令 $u=a_2x+b_2y$,此时 $\frac{du}{dx}=a_2+b_2\frac{dy}{dx}$,$f\left(\frac{k(a_2x+b_2y)+c_1}{a_2x+b_2y+c_2}\right) = f\left(\frac{ku+c_1}{u+c_2}\right)=g(u)$。方程 (\ref{eq:general_linear_fraction_DE}) 化为
    $$\frac{du}{dx} = a_2+b_2g(u),$$
    属于变量分离方程 (\ref{eq:variable_separation_form})。
\end{enumerate}

\end{enumerate}
\end{solution}

\section{常数变易法}\label{sec:常数变易法}

\begin{definition}[一阶线性微分方程]
\begin{equation}
\frac{dy}{dx} = P(x)y+Q(x), \label{eq:first_order_linear_DE_general}
\end{equation}
其中 $P(x), Q(x)$ 在考虑的区间上是 $x$ 的连续函数。若 $Q(x)=0$,式 (\ref{eq:first_order_linear_DE_general}) 变为
\begin{equation}
\frac{dy}{dx} = P(x)y, \label{eq:first_order_linear_DE_homogeneous}
\end{equation}
式 (\ref{eq:first_order_linear_DE_homogeneous}) 称为\textbf{一阶齐次线性微分方程}。若 $Q(x) \ne 0$,式 (\ref{eq:first_order_linear_DE_general}) 称为\textbf{一阶非齐次线性微分方
程}。
\end{definition}

\begin{example}
    求方程
\begin{equation}
    \frac{dy}{dx} = P(x)y \label{eq:example_DE}
\end{equation}
的通解。
\end{example}
\begin{solution}
    $y \ne 0$ 时, 变量分离为
$$\frac{dy}{y} = P(x) dx,$$
两边积分得
$$\ln |y| = \int P(x) dx + \tilde{c},$$
即得通解
$$|y| = e^{\int P(x) dx + \tilde{c}},$$
即
\begin{equation}
y = \pm e^{\tilde{c}} \cdot e^{\int P(x) dx} = c e^{\int P(x) dx}, \label{eq:example_solution}
\end{equation}
这里 $c = \pm e^{\tilde{c}}$。
显然,$y=0$ 也是式 (\ref{eq:example_DE}) 的解。如果式 (\ref{eq:example_solution}) 中允许 $c=0$,则解 $y=0$ 也就包含在式 (\ref{eq:example_solution}) 式
中了。因此,方程的通解为式 (\ref{eq:example_solution}),其中 $c$ 为任意常数。
\end{solution}

\begin{definition}[常数变易法]\label{def:常数变易法}
    将齐次微分方程中常数 $c$ 变为待定函数 $c(x)$ 的方法,称为\textbf{常数变易法}。常
数变易法实际上也是一种变量变换的方法。
\end{definition}
\begin{example}
    讨论非齐次线性微分方程 (\ref{eq:first_order_linear_DE_general}) 通解的求法。
\end{example}
\begin{solution}
    (\ref{eq:first_order_linear_DE_homogeneous}) 是 (\ref{eq:first_order_linear_DE_general}) 的特殊情形,如果我们对 (\ref{eq:first_order_linear_DE_homogeneous}) 也形式求解,可以得到
通解为
\begin{equation}
y = c(x)e^{\int P(x)dx} \label{eq:variation_of_parameters_ansatz}
\end{equation}
的形式。这是在式 (\ref{eq:example_solution}) 中,将常数 $c$ 变成 $x$ 的待定函数 $c(x)$ 所对应的形式。
对式 (\ref{eq:variation_of_parameters_ansatz}) 求导数,得
\begin{equation}
\frac{dy}{dx} = \frac{dc(x)}{dx}e^{\int P(x)dx} + c(x)P(x)e^{\int P(x)dx}. \label{eq:dy_dx_derived}
\end{equation}
将式 (\ref{eq:variation_of_parameters_ansatz}),(\ref{eq:dy_dx_derived}) 代入式 (\ref{eq:first_order_linear_DE_general}),得到
$$\frac{dc(x)}{dx}e^{\int P(x)dx} + c(x)P(x)e^{\int P(x)dx} = c(x)P(x)e^{\int P(x)dx} + Q(x),$$
即
$$\frac{dc(x)}{dx} = Q(x)e^{-\int P(x)dx},$$
积分后得
$$c(x) = \int Q(x)e^{-\int P(x)dx} dx + \tilde{C},$$
这里 $\tilde{C}$ 是任意常数。将上式代入式 (\ref{eq:variation_of_parameters_ansatz}),得到方程 (\ref{eq:first_order_linear_DE_general}) 的通解为
\begin{equation}
y = e^{\int P(x)dx}\left(\int Q(x)e^{-\int P(x)dx} dx + \tilde{C}\right). \label{eq:first_order_linear_DE_solution}
\end{equation}
\end{solution}

\begin{example}
    形如
\begin{equation}
\frac{dy}{dx} = P(x)y+Q(x)y^n \quad (n \ne 0,1 \text{ 为实数}) \label{eq:Bernoulli_DE}
\end{equation}
的方程,称为\textbf{伯努利 (Bernoulli) 微分方程},其中 $P(x),Q(x)$ 在考虑的区间上是 $x$ 的连
续函数。
\end{example}
\begin{solution}
    利用变量变换可将伯努利微分方程化为线性微分方程。事实上,当 $y \ne 0$ 时,用 $y^{-n}$
乘式 (\ref{eq:Bernoulli_DE}) 两边,得到
\begin{equation}
y^{-n}\frac{dy}{dx} = P(x)y^{1-n}+Q(x), \label{eq:Bernoulli_multiplied}
\end{equation}
引入变量变换
\begin{equation}
z=y^{1-n}, \label{eq:Bernoulli_substitution_z}
\end{equation}
从而
\begin{equation}
\frac{dz}{dx} = (1-n)y^{-n}\frac{dy}{dx}. \label{eq:Bernoulli_dz_dx}
\end{equation}
将式 (\ref{eq:Bernoulli_substitution_z}),(\ref{eq:Bernoulli_dz_dx}) 代入式 (\ref{eq:Bernoulli_multiplied}),得到
\begin{equation}
\frac{dz}{dx} = (1-n)P(x)z+(1-n)Q(x), \label{eq:Bernoulli_transformed_linear_DE}
\end{equation}
这是线性微分方程,可按上面介绍的方法求其通解,然后代回原变量即可得到伯努利
微分方程 (\ref{eq:Bernoulli_DE}) 的通解。此外,当 $n>0$ 时,方程还有解 $y=0$。
\end{solution}

\section{恰当微分方程与积分因子}\label{sec:恰当微分方程与积分因子}
\begin{definition}[恰当微分方程]\label{def:恰当微分方程}
\begin{equation}
M(x,y)\,dx+N(x,y)\,dy=0, \label{eq:exact_DE_form}
\end{equation}
这里假设 $M(x,y), N(x,y)$ 在某矩形域内是 $x,y$ 的连续函数,且具有连续的一阶偏导
数。
如果方程 (\ref{eq:exact_DE_form}) 的左端恰好是某个二元函数 $u(x,y)$ 的全微分,即
\begin{equation}
M(x,y)\,dx+N(x,y)\,dy = du(x,y) = \frac{\partial u}{\partial x}dx+\frac{\partial u}{\partial y}dy, \label{eq:exact_DE_condition}
\end{equation}
则称 (\ref{eq:exact_DE_form}) 为\textbf{恰当微分方程}。
\end{definition}
\begin{remark}
    容易验证恰当微分方程的通解为
\begin{equation}
u(x,y)=c, \label{eq:exact_DE_solution}
\end{equation}
这里 $c$ 是任意常数。
\end{remark}

\begin{theorem}[恰当微分方程的判定定理]\label{thm:exact_DE_criterion}
二元函数 $M(x,y), N(x,y)$ 在某单连通域内是 $x,y$ 的连续函数,且具有连
续的一阶偏导数,则方程
$$M(x,y)\,dx+N(x,y)\,dy=0$$
为恰当微分方程的充要条件是
$$\frac{\partial M}{\partial y} = \frac{\partial N}{\partial x}.$$
\end{theorem}

\begin{proposition}[曲线积分的性质]\label{prop:path_independence}
设二元函数 $M(x,y), N(x,y)$ 在某单连通域 $D$ 内是 $x,y$ 的连续函数,且具有
连续的一阶偏导数,则对 $D$ 内任一按段光滑曲线 $L$, 曲线积分 $\int_L M(x,y)\,dx + N(x,y)\,dy$ 在
区域 $D$ 内积分与路径无关的充要条件是 $\frac{\partial M}{\partial y} = \frac{\partial N}{\partial x}$.
\end{proposition}

\begin{example}
    解方程 $(3x^2+6xy^2)dx+(6x^2y+4y^3)dy=0$.
\end{example}
\begin{solution}
    这里 $M=3x^2+6xy^2, N=6x^2y+4y^3$, 有
$$\frac{\partial M}{\partial y} = 12xy, \quad \frac{\partial N}{\partial x} = 12xy.$$
故方程是恰当微分方程。
\begin{enumerate}
    \item 求 $u$ 使其满足
\begin{align}
\frac{\partial u}{\partial x} &= M=3x^2+6xy^2, \label{eq:partial_u_x_M} \\
\frac{\partial u}{\partial y} &= N=6x^2y+4y^3. \label{eq:partial_u_y_N}
\end{align}
将式 (\ref{eq:partial_u_x_M}) 对 $x$ 积分,得
\begin{equation}
u=x^3+3x^2y^2+\varphi(y). \label{eq:u_integrated_x}
\end{equation}
将式 (\ref{eq:u_integrated_x}) 对 $y$ 求导数,并由式 (\ref{eq:partial_u_y_N}) 可得
$$\frac{\partial u}{\partial y} = 6x^2y+\frac{d\varphi(y)}{dy} = 6x^2y+4y^3,$$
于是
$$\frac{d\varphi(y)}{dy} = 4y^3,$$
解得
$$\varphi(y)=y^4.$$
将 $\varphi(y)$ 代入式 (\ref{eq:u_integrated_x}),得到
$$u=x^3+3x^2y^2+y^4.$$
因此,方程的通解为
$$x^3+3x^2y^2+y^4=c,$$
其中 $c$ 为任意常数。
    \item 利用“曲线积分”的方法求解 $u(x,y)$.
由于 $M=3x^2+6xy^2, N=6x^2y+4y^3$, $\frac{\partial M}{\partial y} = 12xy, \frac{\partial N}{\partial x} = 12xy$ 在整个二维实平面上都连
续,因此我们可以利用从原点 $O(0,0)$ 到 $B(x,y)$ 的曲线积分来求解,即
\begin{align*}
u(x,y) &= \int_0^B M(x,y)\,dx +  N(x,y)\,dy \\
&= \int_0^x 3t^2\,dt + \int_0^y (6x^2t+4t^3)\,dt \\
&= x^3+3x^2y^2+y^4.
\end{align*}
即得方程的通解
$$x^3+3x^2y^2+y^4=c,$$
其中 $c$ 为任意常数。
    \item 用“分项组合”方法求解 $u(x,y)$.
将原方程重组:
\begin{align*}
(3x^2+6xy^2)dx+(6x^2y+4y^3)dy &= 3x^2dx+4y^3dy+6xy^2dx+6x^2ydy \\
&= d(x^3)+d(y^4)+d(3x^2y^2) \\
&= d(x^3+y^4+3x^2y^2) = 0.
\end{align*}
即得方程的通解
$$x^3+3x^2y^2+y^4=c,$$
其中 $c$ 为任意常数。
\end{enumerate}

\end{solution}

\begin{definition}[积分因子]\label{def:积分因子}
    如果存在连续可微函数 $\mu(x,y) \ne 0$, 使得
\begin{equation}
\mu(x,y)M(x,y)\,dx+\mu(x,y)N(x,y)\,dy=0 \label{eq:integrating_factor_DE}
\end{equation}
为一恰当微分方程,即存在函数 $u(x,y)$, 使得
$$\mu(x,y)M(x,y)\,dx+\mu(x,y)N(x,y)\,dy=du(x,y)$$
则称 $\mu(x,y)$ 为方程 (\ref{eq:exact_DE_form}) 的\textbf{积分因子}。
这时 $u(x,y)=c$ 是式 (\ref{eq:integrating_factor_DE}) 的通解,因而也是式 (\ref{eq:exact_DE_form}) 的通解。
\end{definition}
\begin{remark}
    一般地可以将带有积分因子的微分方程带入定理 \ref{thm:exact_DE_criterion} 得到积分因子存在的条件,进一步可以假设积分因子只与其中的一个变量有关求解。
\end{remark}

\section{一阶隐式微分方程}\label{sec:一阶隐式微分方程}
\begin{example}
    讨论形如
\begin{equation}
y=f(x,y') \label{eq:y_equals_f_xyprime}
\end{equation}
和
\begin{equation}
x=f(y,y') \label{eq:x_equals_f_yyprime}
\end{equation}的方程,这里假设函数 $f(x,y')$ 有连续的偏导数。
\end{example}
\begin{solution}
    引进参数 $p$,令 $y'=p$. 则方程 (\ref{eq:y_equals_f_xyprime}) 变为
\begin{equation}
y=f(x,p), \label{eq:y_equals_f_xp}
\end{equation}
将式 (\ref{eq:y_equals_f_xp}) 两边对 $x$ 求导数,并将 $y'=p$ 代入,得
\begin{equation}
p = \frac{\partial f}{\partial x} + \frac{\partial f}{\partial p}\frac{dp}{dx}. \label{eq:p_partial_f}
\end{equation}
式 (\ref{eq:p_partial_f}) 是关于 $x,p$ 的一阶微分方程,可按照 \ref{sec:变量分离}-\ref{sec:恰当微分方程与积分因子} 解之。

方程 \eqref{eq:x_equals_f_yyprime} 可作类似变换求解。
\end{solution}

\begin{example}
    讨论形如
\begin{equation}
F(x,y')=0 \label{eq:F_x_yprime_0}
\end{equation}
和
\begin{equation}
F(y,y')=0 \label{eq:F_y_yprime_0}
\end{equation}的方程的解法。
\end{example}
\begin{solution}
    令 $y'=p$, 方程化为 $F(x,p)=0$, 代表 $Oxp$ 平面上的一条曲线。如果把曲
线表示成参数方程形式:
\begin{equation}
\begin{cases} x=\varphi(t), \\ p=\psi(t), \end{cases} \label{eq:param_xyprime_form}
\end{equation}
$t$ 为参数。代入 $y'=p$, 得
$$dy=\psi(t)\varphi'(t)\,dt.$$
两边积分,得
$$y=\int \psi(t)\varphi'(t)\,dt + c.$$
代入式 (\ref{eq:param_xyprime_form}),则得原方程的参数形式的通解为
\begin{equation}
\begin{cases} x=\varphi(t), \\ y=\int \psi(t)\varphi'(t)\,dt + c, \end{cases} \label{eq:param_solution_form}
\end{equation}
$c$ 为任意常数。
方程 \eqref{eq:F_y_yprime_0} 可作类似变换求解。
\end{solution}
